\chapter{Theoretical Framework}

- Cumulos globulares (definiciones, propiedades, formacion y evolucion, poblaciones estelares, etc)

\section{Globular Clusters}

Typical galaxies all around the Universe hold different structures such as stellar systems of between $ 10^{2} $ and $ 10^{6} $ stars which orbit their galactic core . We call these interesting systems star clusters and they are basically divided into two main types: Open Clusters and \textbf{Globular Clusters}.

Globular clusters are very massive stellar systems that can contain from thousands to millions of stars in a nearly spherical distribution spread over a volume of several tens to about 200 light years in diameter. These stellar systems are composed of old stars and they do not contain gas or dust. 

The average star density in a Globular Cluster is about 0.4 stars per cubic parsec. In the dense center of the cluster, the star density can increase from 100 to 1000 per cubic parsec. However, even in the center of clusters, there is still plently of space between the stars

A way of analysing globular clusters is to use Colour-Magnitude diagrams. A colour-magnitude diagram is a plot of the apparent magnitudes of the stars in a cluster against their colour indices. Globular clusters nearly all have very similar colour-magnitude diagrams

Globular clusters revolve about the nucleus of a galaxy on orbits of high eccentricity and high inclination to the galactic plane. About a third of globular clusters are concentrated around the galactic center. A typical cluster has a period of revolution around the order of $ 10^{8} $ years. A cluster spends most of its time far from the center of a galaxy, and so most of them can, and have been discovered in the spaces between galaxies.

\begin{figure}[H]
\centering
\includegraphics[width=12cm]{images/m15.jpg}
\caption{Globular Cluster M15, taken by the Hubble Space Telescope with an exposition time of 900 seconds. Image by NASA}
\end{figure}

To ensure the stability of an isolated cluster, the average speed of its individual stars must not exceed the escape velocity from the cluster. If this occurred, the stars would escape into space, and the cluster would dissipate. If the stellar velocities are low enough to satisfy this condition, then the cluster is gravitationally bound, i.e. the force of gravity is strong enough to keep the member stars from escaping.

Due to clusters moving in various orbits in the Galaxy, they are bound together with gravitational forces that are stronger than the disrupting forces exerted on it by the Galaxy or other nearby stars, and this results in an added condition for the stability of a cluster. Another factor in the stability of clusters is size-the smaller and more compact the cluster, the greater its own gravitational binding force compared with the disrupting forces, and the more chance it has to survive to old age.

Because globular clusters are highly compact systems, they are consequently very stable, and so most globular clusters will probably maintain their identity almost indefinitely.

But even these clusters lose some stars, especially if they have a slow mass. This is due to there always being a few stars in a cluster that move faster than the cluster's average speed.

When a star escapes, it carries with it energy, removing this energy from the cluster as a whole. This eventually results in the cluster developing a tightly bound core surrounded by a rarefied halo of stars-much

In the dense core of a cluster, the stars in it occasionally collide, and some of the debris eventually coalesces. Predictions indicate that this dynamical evolution could lead to the development of a large Black Hole at the cluster's center.At the same time, a few stars in the outer parts of the cluster would continue to escape. The escape rate and dynamical evolution for the rich globular clusters are so slow that the clusters can easily survive for many billions of years, remaining mostly unchanged.

Proxima Centauri, and it is 4.2 light-years, or about 1.3 parsecs. Thus, if we were able to draw a sphere around the Sun with a radius of 1.3 parsecs, it would only contain 2 stars: the Sun and Proxima Centauri. If you were to draw this same sphere in the center of the globular cluster M13, it would contain approximately 10,000 stars.

The first globular cluster discovered, but then taken for a nebula, was M22 in Sagittarius, which was probably discovered by Abraham Ihle in 1665. This discovery was followed by that of southern Omega Centauri (NGC 5139) by Edmond Halley on his 1677 journey to St. Helena. This "nebula" had been known but classified as star since ancient times. Next followed the discovery of M5 in Serpens Caput by Gottfried Kirch in 1702, and that of M13 in Hercules, again by Halley, in 1714. De Chéseaux's list of (21) nebulae of 1746 contains, in addition, two new globular clusters, M71 and M4, while Jean-Dominique Maraldi discovered M15 and M2 in September of this year (1746). Guillaume Legentil possibly or probably discovered NGC 6712 in 1749. Nicholas Louis de Lacaille's catalog of (42) southern "nebula" of 1751-52 contains 8 globular clusters (among them 5 new ones), while Messier's catalog of 110 objects contains a total of 29 globulars, 20 of them new discoveries. Charles Messier was the first to resolve one globular cluster, M4, but still referred to the other 28 of these objects in his catalog as "round nebulae." Thus, in summer 1782, before William Herschel started his comprehensive deep sky survey with large telescopes, there were 34 globular clusters known. Herschel himself discovered 36 new globulars, bringing the number of known globulars to 70. He was the first to resolve virtually all of them into stars, and coined the term "globular cluster" in the discussion adjacent to his second catalog of 1000 deepsky objects (1789).

Radial velocity measurements have revealed that most globulars are moving in highly excentric elliptical orbits that take them far outside the Milky Way; they form a halo of roughly spherical shape which is highly concentrated to the Galactic Center, but reaches out to a distance of several 100,000 light years, much more than the dimension of the Galaxy's disk. As they don't participate in the Galaxy's disk rotation, they can have high relative velocities of several 100 km/sec with respect to our solar system; this is what shows up in the radial velocity measurements. Ninkovic (1983) has estimated excentricities of globular cluster orbits.

To determine the physical orbits of globular clusters, it is required to know their proper motions in addition to the radial velocities. Cudworth and Hanson (1993) undertook some first rough determinations of proper motions with respect to background galaxies. From these and similar measurements, Van den Bergh (1995) estimated perigalactic distances, and Dauphole et.al. (1996) calculated first approximate orbits. Much more acurate data for proper motions became only available from astrometrical data obtained with ESA's Hipparcos satellite in 1997, from which space motions (Geffert et.al. 1997) and approximate orbits (Brosche et.al. 1997) could be determined.

\subsection{Photometric Properties}

The HR diagram for a typical globular cluster looks very different than that of an open cluster. There are no Main Sequence stars of types OBAF, but there are many red giants. The brightest stars in a globular cluster are those at the tip of the red giant branch in the HR diagram, which explains the red appearance of the bright stars in color images of the clusters. You can also see stars populating the horizontal branch (and also why it is called the horizontal branch), the asymptotic giant branch, and even some stars that have colors and magnitudes of F stars, but far fewer than the G stars just below and to the right of them on the Main Sequence.

\begin{figure}[H]
\centering
\includegraphics[width=10cm]{images/m55_diagram.jpg}
\caption{Color Magnitude diagram of M55. Image by NASA}
\end{figure}


\section{Stellar System Dynamics}

(we focus on globular clusters)

\subsection{Colisionless Systems}

\begin{equation}
\dfrac{df}{dt}=0
\end{equation}


\section{Scenario and Observations}

\section{Simulations}

\chapter{Theoretical Framework}

- Cumulos globulares (definiciones, propiedades, formacion y evolucion, poblaciones estelares, etc)

\section{Globular Clusters}

Typical galaxies all around the Universe hold different structures such as stellar systems of between $ 10^{2} $ and $ 10^{6} $ stars which orbit their galactic core. We call these interesting systems star clusters and they are basically divided into two main types: Open Clusters and \textbf{Globular Clusters}.

Globular clusters are very massive stellar systems that can contain from thousands to millions of stars in a nearly spherical distribution. These stellar systems are composed of old stars and they do not contain gas or dust. 

\subsection{Photometric Properties}

\section{Stellar System Dynamics}

(we focus on globular clusters)

\subsection{Colisionless Systems}

\begin{equation}
\dfrac{df}{dt}=0
\end{equation}


\section{Scenario and Observations}

\section{Simulations}

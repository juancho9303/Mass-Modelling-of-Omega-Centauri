\chapter{Introduction}

Despite the large amount of observational data and the advances in theoretical work, still, there is not yet a convincing model to describe the formation of ancient globular clusters. In the current scenario of galaxy formation, galaxies are formed in to the deepest regions of the gravitational potential well provided by dark matter halos. Peebles (1984) proposed such an scenario would be also valid to explain the formation of ancient globular clusters. This explanation has been received with great interest by the community since it not only fits within the hierarchical scenario of structure and galaxy formation, but also may help to understand some of the open problems in the standard model, such as the abundance of low mass structures (Klypin et. al. 1999). However, evidence has been found against this scenario. For example, Odenkirchen et al. (2003) have found tidal tails surrounding globular clusters, something that is not expected if globular clusters form and reside inside extended dark matter halos.

On the other hand, Fall \& Ries (1985) propose the formation of globular clusters as a response to instabilities in the hot gaseous halos of massive galaxies. This scenario presents a problem since there are observations that suggest that very low mass galaxies, not massive enough to host a hot gaseous halo, may also have their own globular clusters.

Conroy et. al. (2011) have used density profiles to argue against the presence of dark matter inside globular clusters, while Ibata et. al. (2013) have found that under general conditions it could be possible to find significant fractions of dark matter in globular clusters. Modelling the mass content of globular clusters in the galaxy would help to disentangle the mechanism driving its formation process. Mass modelling would allow to study the mass distribution in the
inner region of globular clusters, determining the dominant components (Breddels M. A. et.al. 2013, Adams J. J. et. al., 2012, van den Bosch R.C.E. et. al. 2006) providing light on the problem of the origin of globular clusters.

Our aim in this project is to build our own mass model for the Globular Clusters using some data observed in OPD observatory in May 2014 so that we can be able to discuss and find new evidence on the existence or absence of dark matter in them according to these results. 

With our set of data, we do the preliminary reduction and analysis of photometric and spectroscipic data, including the wavelength and flux calibration of the spectra. The photometric data will show us the mass to light ratio of the Globular Clusters thus giving us information about the mass content and distribution in the clusters. The spectroscopic data will give us information about the velocity dispersion of the stars in the inner region of the clusters thus giving us information about the potential well in the clusters. This procedure is done using the Radial Velocity Package of IRAF called \textit{RVSAO} which uses cross correlation techniques in the Fourier transform of template and scientific spectra to infer the redshift of the emission and/or absorption lines in our data and calculating the radial velocities. 

After this analysis has been made upon all the observational data we can start the theoretical analysis including simulations of N-body and study of the initial conditions that will preserve the spherical symmetry of the clusters. The dynamics and mass modelling is made in the context of the Jeans equations and the collisioneless Bolztmann equation since these are the most general equations for mass and dynamics modelling of galaxies.



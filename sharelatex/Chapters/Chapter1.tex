\chapter{Introduction}

As the oldest known subunits of our Galaxy, Globular Clusters are the primary "fossils" from its early evolution, and they may hold the key to understanding its formation. For example, the amount of dark matter that they could (or not) have may be an important clue to the understanding of the formation of the Milky Way. Despite the large amount of observational data and the advances in theoretical work, still, there is not yet a convincing model to describe the formation of these ancient structures. 

Two broad types of possibilities have been considered: one is that the globular clusters were the very first condensed systems to form in the early universe, and the second is that they originated in larger star-forming systems that later merged to form the present galaxies. 

The first possibility includes the suggestion of Peebles \& Dicke (1968) that the globular clusters were formed by Jeans fragmentation in the early universe, this possibility is in concordance with the current scenario of galaxy formation in which galaxies are formed inside the deepest regions of the gravitational potential well provided by dark matter halos. This explanation has been received with great interest by the scientific community since it not only fits within the hierarchical scenario of structure and galaxy formation, but also may help to understand some of the open problems in the standard model, such as the abundance of low mass structures (Klypin et. al. 1999). However, evidence has been found against this scenario. For example, Odenkirchen et al. (2003) have found tidal tails surrounding globular clusters, something that is not expected if globular clusters form and reside inside extended dark matter halos.

Regarding the second possibility, Fall \& Ries (1985, 1988) proposed the formation of globular clusters as a response to thermal instabilities in the hot gaseous halos of massive galaxies. The Fall \& Rees hypothesis has been popular among theorists who have used it to predict the characteristic properties of globular clusters, although it has proven difficulties to justify the assumed thermal behaviour of the cluster-forming gas clouds. This scenario presents a problem since there are observations that suggest that very low mass galaxies, not massive enough to host a hot gaseous halo, may also have their own globular clusters.

Not only the formation of these structures is puzzling, their composition is also a challenging problem because some authors say that these structures do not contain any dark matter contributions and that their gravitational stability can be explained completely with the baryonic matter inside of them. Conroy et. al. (2011) have used density profiles to argue against the presence of dark matter inside globular clusters, while Ibata et. al. (2013) have found that under general conditions it could be possible to find significant fractions of dark matter in globular clusters. Modelling the mass content of globular clusters in the galaxy would help to disentangle the mechanism driving its formation process. Mass modelling would allow to study the mass distribution in the inner region of globular clusters, determining the dominant components (Breddels M. A. et.al. 2013, Adams J. J. et. al., 2012, van den Bosch R.C.E. et. al. 2006) providing light on the problem of the origin of globular clusters.

Our aim in this project is to build our own mass model for the Globular Clusters using some data observed in OPD observatory in May 2014 so that we can be able to discuss and find new evidence on the existence or absence of dark matter in them according to these results. 

With our set of data, we do the preliminary reduction and analysis of photometric and spectroscopic data, including the wavelength and flux calibration of the spectra. The photometric data will show us the mass to light ratio of the Globular Clusters thus giving us information about the baryonic mass content of the clusters. On the other hand, the spectroscopic data will give us information about the radial velocity of the stars that will provide us the statistical dispersion of velocities in the inner region of the clusters thus giving us information about the potential well in the clusters. This procedure is done using the Radial Velocity Package of IRAF called \textit{RVSAO} which uses cross correlation techniques in the Fourier transform of templates and scientific spectra to infer the doppler broadening of the emission and/or absorption lines in our data and allows us to calculate their radial velocities. 

After this analysis has been made upon all the observational data we can start the theoretical analysis including simulations of N-body systems and study of the initial conditions that will preserve the spherical symmetry of the clusters, taking into account our results on the dominant components of the clusters. 

The bulk properties of GCs, with the possible exception of their innermost regions, can be modelled using the collisionless Boltzmann equation (Binney \& Tremaine 1987), from which the statistical properties of the velocity distribution of stars can be derived. In particular, one can derive a formulation for the velocity dispersion tensor that, in the isotropic, non rotating case, reduces to a scalar quantity. This quantity can be determined using the appropriate Jeans equation. This calculation requires knowledge of the distribution function (DF) that determines the number of stars in a given region of space, and the gravitational potential.

The best fit for our data with the theoretical assumptions will tell us how mass is distributed in the clusters and it will allow us to conclude if there is a significant contribution of dark matter to the potential well that provides the stars the observed velocities. As mentioned before, the Mass Modelling will let us have a good insight into the problem of the formation of these structures.




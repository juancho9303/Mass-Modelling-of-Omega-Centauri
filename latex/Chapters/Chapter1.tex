\chapter{Introduction}

Old Globular Clusters are among the oldest known subunits of galaxies, so that they might hold the key of understanding the galaxies' very first evolution stages and their formation. In particular, the amount of dark matter that they could (or not) have could provide valuable information to the current theories about the creation and early development of the Milky Way (and other galaxies in general). Even though there is a large amount of observational data that has been obtained since the late 18$^{th}$ century, and many advances in theoretical work involving their photometric evolutionary properties, their trajectories in the galactic halo and star formation, there is not yet an accepted model to describe the early formation of these ancient structures nor a clear understanding of their mass content. In addition, recent observations show that their stellar populations are far more complex than initially thought (R. G. Gratton et. al. 2012), an issue that is directly connected to the formation process itself. 

In order to access the problem of their formation, many theories have been considered, although there are two that stand out the most: One states that old Globular Clusters were the very first condensed systems to form in the early universe, by the time big structures such as galaxies were only about to start their formation. The second possibility is that they originated in larger star-forming systems that later merged to form the present galaxies. These two broad possibilities are discussed in detail by Larson (1996).

The first possibility (suggested by Peebles \& Dicke in 1968) says that Globular Clusters (GCs) were formed by a process of Jeans instability in the early universe, they suggest that the smallest gravitationally unstable clouds that were produced right after the recombination from isothermal perturbations were the progenitors of GCs, this possibility is in concordance with the current scenario of galaxy formation in which galaxies are formed inside the deepest regions of the gravitational potential well provided by dark matter halos and it has been received with great interest by the scientific community since it not only fits within the hierarchical scenario of structure and galaxy formation, but also may help to understand some of the open problems such as the missing satellites around galaxies like ours (Klypin et. al. 1999). However, evidence has been found against this scenario. One is the fact that this early attempt did not take into account the issues of the multiple stellar generations that have been recently found in these stellar systems. Another problem is the findings of tidal tails surrounding GCs, which is not expected if these structures are formed and reside inside their own extended dark matter halos (Odenkirchen et al. (2003)).

Regarding the second possibility, Fall \& Rees (1985, 1988) proposed that the formation of globular clusters was the product of the collapsing gas of a protogalaxy. The mechanism would be the response to thermal instabilities in the hot gaseous halos of massive galaxies, this hypothesis starts from the assumption that star formation can only occur when the gas has been able to cool in a free-fall time. The Fall \& Rees hypothesis has been popular among theorists who have used it to predict the characteristic properties of globular clusters such as its poor gas content, although it has proven difficulties to justify the assumed thermal behaviour of the cluster-forming gas clouds in many regions of the galaxy. This scenario presents other problems since there are observations such as those made by Sharina et. al. 2005 that suggest that very low mass galaxies, not massive enough to host a hot gaseous halo, may also have their own GCs , so that their formation might be the result of a different mechanism.

Not only the formation of these structures is puzzling, their composition is also a challenging problem because some authors argue that these structures do not contain any dark matter contributions and that their dynamics can be explained completely with the baryonic matter inside them. For example, Conroy et. al. (2011) used density profiles to argue against the presence of dark matter inside globular clusters by demonstrating that the outer stellar density profile of isolated GCs is
very sensitive to the presence of an extended dark halo which is not seen in the observations of NGC 2419; while Ibata et. al. (2013) have found that under general conditions it could be possible to find significant fractions of dark matter in globular clusters, by using a Markov-Chain Monte Carlo approach and modelling current observations. Modelling the mass content of GCs in the galaxy would help to disentangle the mechanism driving its formation process and would give us some insight into their stellar populations as well. Detailed mass modelling would allow us to study the mass distribution in the inner region of globular clusters, giving us information about the dominant components, just like Adams J. J. et. al., 2012 did with NGC2976. Providing light on the problem of the origin of globular clusters.

Our aim in this thesis is to build our own mass model for the globular Cluster many other databases so that we can be able to discuss and find new evidence on the existence or absence of dark matter in the cluster and argue about the mentioned sets of possibilities about GC's formation. We also use our results to analyse the growing theory that $\omega$ Centauri is actually the remnants of the nucleus of a dwarf galaxy absorbed by the Milky Way (Toshio et al. 2003, Norris 2012). These procedures are in principle applicable to all globular clusters with similar characteristics of $\omega$ Cen. 

With our set of data, and mostly for pedagogical purposes due to some instrumental problems, we do the reduction and extraction procedures for the photometric and spectral data (including the wavelength and flux calibration of the spectra). The photometric data from databases will show us the mass to light ratio of the Globular Clusters thus giving us information about the baryonic mass content of the clusters. On the other hand, the spectroscopic data will give us information about the radial velocity of the stars that will provide us the statistical dispersion of velocities in the inner region of the clusters thus giving us information about the potential well in the clusters.  This procedure is done using the Radial Velocity Package of IRAF called \textit{RVSAO} which uses cross correlation techniques in the Fourier transform of templates and scientific spectra to infer the doppler broadening of the emission and/or absorption lines in our data and allows us to calculate their radial velocities. The task XCSAO is also used by some authors who kindly provided their data via Vizier. Our spectroscopic data will also give us information about the stellar populations of the clusters so that we can also infer the baryonic mass using this technique. 

The bulk properties of the cluster, with the possible exception of their innermost regions, can be modelled using the collisionless Boltzmann equation (Binney \& Tremaine 1987), from which the statistical properties of the velocity distribution of stars can be derived. In particular, one can derive a formulation for the velocity dispersion tensor that, in the isotropic, non rotating case, reduces to a scalar quantity (as discussed in further detail in the next chapter). This quantity can be determined using the appropriate Jeans equations. Most of these calculations require knowledge of the distribution function (DF) that determines the number of stars in a given region of space, and the gravitational potential.

The best fit for our data with the theoretical assumptions will tell us how mass is distributed in the cluster and it will allow us to conclude if there is a significant contribution of dark matter to the potential well that provides the stars the observed velocities. As mentioned before, the Mass Modelling will let us have a good insight into the problem of the formation of these structure.

\newpage

\chapter{Modelling}

We used various techniques for the mass modelling of $\omega$ Centauri, so that we could make a good approximation to its dynamic and stellar mass. In order to do this, we decided to use two components (the stellar and dark matter mass) both following the functional form of the Hernquist profile.  

\section{Hernquist Model}

Our modelling is based on the Hernquist profile (Hernquist 1989).  We use this model because it is a well known model that closely approximates the de Vaucouleurs law for elliptical galaxies and has analytical solutions that can be useful for our computational purposes. As introduced in chapter 2, the density profile for this model is

\begin{equation}
\rho(r)=\frac{M}{2\pi}\frac{a}{r}\frac{1}{\left(r+a\right)^{3}}
\end{equation}

Where $a$ and $M$ are the scalength and total mass associated to the profile that, as we will show, can be associated to the dark or baryonic matter. The profile's cumulative mass is

\begin{equation}
M(r)=M\frac{r^{2}}{(r+a)^{2}}
\end{equation}
 
And the surface brightness 
 
 \begin{equation}
 I(R)=\frac{M}{2\pi a^{2}\Gamma\left(1-s^{2}\right)^{2}}\left[\left(2+s^{2}\right)X(s)-3\right]
 \end{equation}
 
where $s=R/a$, $R$ is the projected radius and:

\begin{equation}
X(s)=\frac{1}{\sqrt{1-s^{2}}}\text{sech}^{-1}s \qquad for\qquad0\leq s\leq1
\end{equation}

\begin{equation}
X(s)=\frac{1}{\sqrt{s^{2}-1}}\sec^{-1}s\qquad for\qquad1\leq s<\infty
\end{equation}

For computational simplicity we have written some of the trigonometric functions just like Hernquist did: $\sec^{-1}s = \cos^{-1}(1/s)$ and $\text{sech}^{-1} = \ln[(1+\sqrt{1-s^{2}})/s]$. As mentioned in the theoretical framework chapter, the line of sight velocity dispersion in the more general case with the anisotropy parameter different from 0 we have

\begin{equation}
I(R)\sigma_{p}^{2}(R)=\frac{2}{\Gamma}\int_{R}^{\infty}\left(1-\beta\frac{R^{2}}{r^{2}}\right)\frac{\rho\bar{v_{r}^{2}}rdr}{\sqrt{r^{2}-R^{2}}}
\end{equation}

And the radial velocity dispersion (introduced in chapter 2), in terms of the potential and the density is

\begin{equation}
\bar{v_{r}^{2}}=\sigma_{r}^{2}=\frac{1}{\rho(r)}\int_{r}^{\infty}\rho(r)\frac{d\phi}{dr}dr
\end{equation}

Where 

\begin{equation}
\frac{d\phi}{dr}=\frac{GM(r)}{r^{2}}
\end{equation}

So the projected velocity dispersion becomes:

\begin{equation}
\sigma_{p}^{2}(R)=\frac{2G}{I(R)\Gamma}\int_{R}^{\infty}\left(1-\beta\frac{R^{2}}{r^{2}}\right)\left(\int_{r}^{\infty}\frac{\rho(r)M(r)}{r^{2}}dr\right)\frac{rdr}{\sqrt{r^{2}-R^{2}}}
\end{equation}

We do various experiments for our modelling, the simplest of all models is the one where we assume that the cluster has just one mass component, in this case there only will be one mass contribution and only one scalength (the stellar scalength) so the solution of the last equation would be:

\begin{equation}
\sigma_{p}^{2}(R)=\frac{GM^{2}a}{I(R)\Gamma\pi}\int_{R}^{\infty}\alpha(r)\left(\frac{\log{\left(\frac{a+r}{r}\right)}}{a^{5}}-\frac{25a^{3}+52a^{2}r+42ar^{2}+12r^{3}}{12a^{4}\left(a+r\right)^{4}}\right)dr
\end{equation}

Where $a$ is the scalength and where, in order to shorten the equation we take $\alpha(r)$ as 

\begin{equation}
\alpha(r)=\left(1-\beta\frac{R^{2}}{r^{2}}\right)\frac{r}{\sqrt{r^{2}-R^{2}}}
\end{equation}

Now, as we want to focus on the dark matter content of the cluster, we assume that the mass of the cluster is the sum of the mass of stars and the mass of non-baryonic matter so that their contributions to the density and mass profiles become:

\begin{equation}
\rho(r)=\rho_{s}(r)+\rho_{dm}(r)\qquad and \qquad M(r)=M_{s}(r)+M_{dm}(r)
\end{equation} 

In this case, the projected velocity dispersion takes a much more complicated form as follows:

\begin{equation}
\begin{aligned}	
\sigma_{p}^{2}(R) &= \frac{G}{I(R)\Gamma\pi}\int_{R}^{\infty}\alpha(r)\Biggl[\underbrace{\int_{r}^{\infty}\frac{M_{s}^{2}a_{s}dr}{r\left(r+a_{s}\right)^{5}}}_{\mathbf{A}(r)}+\underbrace{\int_{r}^{\infty}\frac{M_{s}M_{dm}a_{s}dr}{r\left(r+a_{s}\right)^{3}\left(r+a_{dm}\right)^{2}}}_{\mathbf{B}(r)}\right\\     &+ \underbrace{\int_{r}^{\infty}\frac{M_{dm}M_{s}a_{dm}dr}{r\left(r+a_{dm}\right)^{3}\left(r+a_{s}\right)^{2}}}_{\mathbf{C}(r)}+\underbrace{\int_{r}^{\infty}\frac{M_{dm}^{2}a_{dm}dr}{r\left(r+a_{dm}\right)^{5}}}_{\mathbf{D}(r)}\Biggr] dr
\end{aligned}
\end{equation}

The functional form of the density involves the use of a stellar scalength ($a_{s}$), a dark matter scalength ($a_{dm}$), a stellar mass ($M_{s}$) and a dark matter mass ($M_{dm}$). Now, the integrals $\mathbf{A}(r),\mathbf{B}(r),\mathbf{C}(r)$ and $\mathbf{D}(r)$ have the following analytical solutions:

\begin{equation}
\textbf{A}(r)=a_{s}\left(-\frac{25a_{s}^{3}+52a_{s}^{2}r+42a_{s}r^{2}+12r^{3}}{12a_{s}^{4}\left(a_{s}+r\right)^{4}}+\frac{\log{\left[\frac{a_{s}+r}{r}\right]}}{a_{s}^{5}}\right)
\end{equation}

\begin{equation}
\textbf{D}(r)=a_{dm}\left(-\frac{25a_{dm}^{3}+52a_{dm}^{2}r+42a_{dm}r^{2}+12r^{3}}{12a_{dm}^{4}\left(a_{dm}+r\right)^{4}}+\frac{\log{\left[\frac{a_{dm}+r}{r}\right]}}{a_{dm}^{5}}\right)
\end{equation}

\begin{equation}
\textbf{B}(r)=\frac{\left(M_{s}M_{dm}\right)\left(\mathbf{b_{2}}+\mathbf{b_{3}}+a_{s}\left(-\left(a_{s}-a_{dm}\right)a_{dm}\mathbf{b_{4}}+\mathbf{b_{5}}\right)\right)}{\mathbf{b_{1}}}
\end{equation}

\begin{equation}
With \left\lbrace
\begin{array}{lllll}
\mathbf{b_{1}}=2a_{s}^{2}(a_{s}-a_{dm})^{4}a_{dm}^{2}(a_{s}+r)^{2}(a_{dm}+r)\\
\mathbf{b_{2}}=-2\left(a_{s}-a_{dm}\right)^{4}\left(a_{s}+r\right)^{2}\left(a_{dm}+r\right)\log{r}\\
\mathbf{b_{3}}=2a_{dm}^{2}\left(6a_{s}^{2}-4a_{s}a_{dm}+a_{dm}^{2}\right)
\left(a_{s}+r\right)^{2}\left(a_{dm}+r\right)\log{[a_{s}+r]}\\
\begin{aligned}	
\mathbf{b_{4}} &= 2a_{s}^{4}+4a_{s}^{3}r-2a_{dm}r(a_{dm}+r)+3a_{s}a_{dm}\left(-a_{dm}^{2}+a_{dm}r+2r^{2}\right)\\      &+a_{s}^{2}\left(7a_{dm}^{2}+7a_{dm}r+2r^{2}\right)
\end{aligned}\\
\mathbf{b_{5}}=2a_{s}^{2}\left(a_{s}-4a_{dm}\right)\left(a_{s}+r\right)^{2}\left(a_{dm}+r\right)\log{[a_{dm}+r]}
\end{array}
\right.
\end{equation} 

And

\begin{equation}
\textbf{C}(r)=\frac{\left(M_{dm}M_{s}\right)\left(\mathbf{c_{2}}+\mathbf{c_{3}}+a_{dm}\left(-\left(a_{dm}-a_{s}\right)a_{s}\mathbf{c_{4}}+\mathbf{c_{5}}\right)\right)}{\mathbf{c_{1}}}
\end{equation}

\begin{equation}
With \left\lbrace
\begin{array}{lllll}
\mathbf{c_{1}}=2a_{dm}^{2}(a_{dm}-a_{s})^{4}a_{s}^{2}(a_{dm}+r)^{2}(a_{s}+r)\\
\mathbf{c_{2}}=-2\left(a_{dm}-a_{s}\right)^{4}\left(a_{dm}+r\right)^{2}\left(a_{s}+r\right)\log{r}\\
\mathbf{c_{3}}=2a_{s}^{2}\left(6a_{dm}^{2}-4a_{dm}a_{s}+a_{s}^{2}\right)\left(a_{dm}+r\right)^{2}
\left(a_{s}+r\right)\log{[a_{dm}+r]}\\
\begin{aligned}	
\mathbf{c_{4}} &= 2a_{dm}^{4}+4a_{dm}^{3}r-2a_{s}r(a_{s}+r)+3a_{dm}a_{s}\left(-a_{s}^{2}+a_{s}r+2r^{2}\right)\\      &+a_{dm}^{2}\left(7a_{s}^{2}+7a_{s}r+2r^{2}\right)
\end{aligned}\\
\mathbf{c_{5}} =2a_{dm}^{2}\left(a_{dm}-4a_{s}\right)\left(a_{dm}+r\right)^{2}\left(a_{s}+r\right)\log{[a_{s}+r]}
\end{array}
\right.
\end{equation} 

As we mentioned in the last chapter, we used many databases for projected radial velocities in order to properly do our fit and modelling. The radial velocities as a function of the projected radius is shown in figure 4.1

\begin{figure}[H]
\centering
\includegraphics[width=11cm]{images/vel_vs_rad.png}
\caption[Radial velocity database]{Radial velocities vs projected radius in arcmin. The data is taken from all the databases that gave us a good level of confidence, note that all the data have their respective error.}
\end{figure}

Because we need to calculate the projected velocity dispersion profile, we need to cut in radial bins as shown in figure 4.2. 

\begin{figure}[H]
\centering
\includegraphics[width=11cm]{images/vel_vs_rad_bins.png}
\caption[Radial bins used to calculate velocity dispersions]{Radial bins that we use to calculate the radial projected velocity dispersion in the cluster using equation 4.20}
\end{figure}

The velocity dispersion is estimated as the standard deviation of the velocity in each radial bin because we want to see how much the velocity data deviates from the median, and it is calculated with the formula

\begin{equation}
\sigma = \sqrt{\frac{\sum_{i=1}^{n}\left(v_{i}-v\right)^{2}}{N}}
\end{equation}

To see if the estimation is well made, we plot histograms for each bin to see if the velocities follow a Gaussian behaviour, and where the number of bars in each histogram is given by the square root of the number of velocities in each bin. For some of the radial bins of our profile, the histograms are shown in figure 4.3

\begin{figure}[H]
\centering
\includegraphics[width=15cm]{images/bines.png}
\caption[Histograms for some of the radial bins in our profile]{The histograms for the $5^{th}$, $6^{th}$, $7^{th}$ and $8^{th}$ radial bins. Note that the frequency of repetition of the velocities displayed in bars follows a Gaussian behaviour which suggests that the method used to calculate the velocity dispersion in accurate enough for the modelling.}
\end{figure}

In order to do a proper modelling and fitting we need to take into account the error associated with $\sigma_{p}$. Because all the databases provided the associated error to the measurements of the radial velocities, we made the proper calculation of the error of the velocity dispersion using the general error propagation formula

\begin{equation}
\sigma(v\pm\Delta v)\approx \sigma(v)\pm \underbrace{\frac{\partial \sigma}{\partial v}\Delta v}_{\Delta \sigma}
\end{equation}

So the error of the projected velocity dispersion is estimated as

\begin{equation}
\Delta \sigma = \frac{1}{\sqrt{N}}\left(\sum_{i=1}^{n}\left(v_{i}-v\right)^{2}\right)^{-1/2}\sum_{i=1}^{n}\left(v_{i}-v\right)\Delta v
\end{equation}

With the radial projected velocity dispersion (and its error) we can start with the fitting to find the optimized parameters in our various experiments taking into account several physical constraints. 

\section{Setting constraints}

\subsection{Stellar mass}

The stellar mass content of Globular Clusters and Galaxies can be studied through the determination of the stellar populations inside those systems because we have clear knowledge about how their photometric properties relate to their mass. If we have information about the relative abundance of the stellar populations inside the stellar system, then we can infer the total mass of the cluster by summing up all of these contributions. 

The determination of the stellar populations can be done using STARLIGHT, which is a Fortran-based program that fits an observed integrated spectrum (the central region of Omega Centauri in our case) with a model spectrum which is the sum of $N_{*}$ spectral components from a pre-defined and pre-processed set of base spectra. The program does as many iterations as the user decides to sum up the different template spectra until a good fitting of the spectral lines has been made to the observed spectrum. 

The output of the program after the execution contains the created spectrum (wavelength and intensity) and the approximate percentage of each of the stellar populations inside the stellar system that we use for the determination of the total mass of the cluster.

First, one must prepare the observed spectrum before running STARLIGHT, the spectrum has to be wavelength and flux calibrated, taking into account the bad-pixel removal. Very importantly in the context of mass analysis, the spectrum has to be extinction corrected so that the units of flux relate properly to the units of the templates in STARLIGHT.     

The extinction correction for our observed spectrum is given by

\begin{equation}
f_{obs}(\lambda)=f_{int}(\lambda)10^{-0.4A_{\lambda}}
\end{equation}

Where $A_{\lambda}=0.213$ in the I filter around $8000 \textrm{\AA}$, around the wavelength range of our spectrum. 

In our case, we have to multiply by a factor of 1.216746 the intensity of the spectrum to correct of extinction. After we apply the extinction correction to the spectrum and create an ASCII table with the wavelength, intensity and error columns, it is now ready to be processed with STARLIGHT as we can see in the figure 4.9.

\begin{figure}[H]
\centering
\includegraphics[width=10cm]{images/extinction.png}
\caption[Extinction Correction]{This figure shows an integrated spectrum of the central region of Omega Centauri before and after the extinction correction is applied. The black line has the original flux values and the black line has the corrected flux, that is, the flux that would be observed if there wasn't any interstellar medium that obscures the light coming from the object.}
\end{figure}

Before running STARLIGHT one must assure that the wavelength range is correctly specified in the configuration file that also includes the database of the template spectra and the bad data organized in a mask file. When all of these are ready it is straightforward to run STARLIGHT with the following command (in a linux-based computer):

\begin{center}
./StarlightChains\_v04.exe $<$ Omega\_cen.in
\end{center}

The synthetic spectrum and the original one are shown in figure 4.10.

\begin{figure}[H]
\centering
\includegraphics[width=10cm]{images/comparison.png}
\caption[Synthetic spectrum of STARLIGHT]{Synthetic spectrum of Starlight in red, shifted in the y axis for doing the comparison with the original spectrum of Omega Centauri in blue. Note how similar the spectra are, meaining that the results given by Starlight were very accurate}
\end{figure}
 
Now, besides the synthetic spectrum, the output file contains some useful results that one can use to estimate the mass of the stellar system. In our case, the relevant parameter that STARLIGHT gives is the ``stellar mass parameter":

\begin{equation}
M_{cor\_tot} = 3.29446 \times 10^{7} M_{\odot}/cm^{2}
\end{equation}

And using the formula for the total mass in units of solar masses:

\begin{equation}
M_{s}(M_{\odot})=M_{cor\_tot}\times10^{-17}\times4\pi d^{2}\times\left(3.826\times10^{33}\right)^{-1}
\end{equation}

Where $d$ is the luminosity distance in cm. This equation yields a stellar mass of $M_{\star}=243.462M_{\odot}$ but it is only the mass estimated from the integrated spectrum of the detector area in the cluster, that as shown in Figure 4.6 is only a portion of the whole cluster.

\begin{figure}[H]
\centering
\includegraphics[width=12cm]{images/OmegaCentauriii.jpg}
\caption[Illustration of the detected area of the integrated spectrum]{Illustration of the detected area of the integrated spectrum of $\omega$ Centauri. The marked region represents the detected area in a bigger scale than it truly is, and has the angular dimensions shown in the figure. The dotted line represents the effective radius of the cluster (also in scale) that we use to extrapolate the results found for the detected area and make an approximation to what the whole stellar mass would be if we had an integrated spectrum of the whole globular cluster.}
\end{figure}

As this mass is the stellar mass contained only in the detection area (that in our set up configuration in OPD ends up to be $A_{D}=0.36\,pc^{2}$) we must extrapolate this result to its whole effective area to calculate the whole stellar mass of the system, noting that this will increase the error of the calculation.

If we take the cluster's tidal radius of 40' and it's distance to the sun of $4808.39\,pc$ using a distance modulus of 13.41, then the total effective area (where the stellar mass could be calculated using stellar population synthesis) is $A_{OC}=9833.8\,pc^{2}$. 

Finally, the total stellar mass of the Cluster using this technique can be calculation using:

\begin{equation}
M_{s T} = N \times M_{\star}
\end{equation}

Where N is the number of detection areas within the total effective area of Omega Centauri ($A_{OC}/A_{D}$) of about 31844.8. So that our calculation of the stellar mass is finally:

\begin{equation}
M_{s T} = 6.61 \times 10^{6}M_{\odot}
\end{equation}
 
This result is actually larger than some values  of the dynamical mass found in the literature and it should in principle, be smaller or at least equal to the dynamical mass, this discrepancy in our first approach to the mass determination is probably due to errors given by the extrapolation of the results of the detection area to the whole area of the cluster, because our detection area was very small ($\sim 0.2 \, arcmin^{2}$) compared to the cluster's size of more than $6,000 \, arcmin^{2}$. Still, the stellar population technique is consistent with the order of magnitude of the cluster's mass previously reported. 

\subsection{Stellar scalength}

Because we wanted to be as accurate as possible in our modelling we found a way to calculate the stellar scalength and use is as a constraint to our modelling. We found the value of $a_{s}$ by fitting the effective radius in a de Vaucouleurs profile on the surface brightness of Eva Noyola's observational data of Omega Centauri (Noyola et al. 2013) shown in figure 4.7

\begin{figure}[H]
\centering
\includegraphics[width=11cm]{images/noyola.png}
\caption[Surface brightness profile of Omega Centauri]{Surface brightness profile for $\omega$ Centauri. The circles show Noyola's measured photometric points. The triangles show photometric points obtained from ground based images by Trager et al. The dashed line is Trager’s Chebychev fit. The solid line is Noyola's smooth fit that we don't use for our fitting of the scalength. Figure taken from Noyola et al. 2013}
\end{figure}

We fit the de Vaucouleurs profile (4.49) which is in terms of the surface brightness to this observational data 

\begin{equation}
\mu(r)=\mu_{e}+8.32678\left[\left(\frac{r}{R_{e}}\right)^{1/4}-1\right]
\end{equation}

The fitting gives an effective radius of $R_{e}=4.048pc$. We then use the relation $R_{e}\approx1.8153a$ (Hernquist 1989) that allows us to find a stellar scalength of our Hernquist profile of $a_{s}=2.23pc$. This is one of the constraints we use for some of our experiments.

\subsection{Central region}

Some authors (Jalali et al. 2012, Anderson \& Roeland. 2006) state that there might be a medium-mass black hole at the center of $\omega$ Centauri, so we need to take this into account if we want to make a proper modelling that is not affected by the presence of such a massive body in the velocity dispersion profile of the system.

In some of our experiments, we will include the central region of the cluster (0.5 to 2.0 arcmin) and in the other set of experiments we will exclude these regions. In terms of the estimated bins, it means that some of the experiments will use the two first radial bins and the other experiments won't. 
Note that in the above figures, there is a 
\section{Procedures}

For the mass modelling of the cluster, we do various experiments for studying the mass distribution of the cluster using the solutions to equations 4.10 and 4.13 taking into account the mentioned constraints. In order to do so, we run 24 different experiments that consist of the variation of every parameter of our modelling (that is, dark matter mass $M_{dm}$, stellar mass $M_{s}$, dark matter scalength $a_{dm}$, stellar scalength $a_{s}$, mass-to-light ratio $\Gamma$ and parameter of anisotropy $\beta$) depending on the conditions that we set for each experiment individually. Our procedures follow the following order:

First, we constraint half of our experiments to the total velocity dispersion profile including the center of the cluster (0.5 to 45 arcmin) and we call this set of experiments ``With Central Region"; the other half of the experiments is constrained to only the outer half of the cluster (2.0 to 45 arcmin) and we call them ``Without Central Region". Next, we use the constraints given by the mass to light ratio found in the literature in a series of experiments that we call ``Fixed $\Gamma$". Also we use the constraint of the stellar mass (found with Starlight) $M_s$ that we call ``Fixed $M_s$" and finally a set of experiments that don't have any of this constraints that we call ``Full".

Now, for each of this sets of experiments we apply some other constraints that are shown as follows: 

\textbf{\textit{i)}} In this one, we don't use the crossed terms (B(r) and C(r)) in equation 4.13, because we wanted to see how relevant they were in the modelling, we call this ``\textbf{\textit{No crossed terms}}". 

\textbf{\textit{ii)}} We assume a constant stellar scalength, we call this ``\textbf{\textit{Fixed $a_{s}$}}", and we use the value of 2,23 found with the surface brightness technique. 

\textbf{\textit{iii)}} We don't assume any of our parameters to be constant, that is, we vary $a_{s}$, $a_{dm}$, $\gamma$, $\beta$, $M_{dm}$ and $M_{s}$ in the parameter matrix, we call this ``\textbf{\textit{Full}}". 

\textbf{\textit{iv)}} We assume the cluster doesn't have any dark matter so the solution of the equation 4.9 is much simpler and it's given by 4.10, we call this ``\textbf{\textit{No Dark matter}}".

The following table summarizes the experiments that we do and the order in which they were respectively done

\begin{table}[H]
\centering
\label{my-label}
\begin{tabular}{|c|c|c|c|c|c|c|c|c|}
\hline
\multicolumn{1}{|c|}{\multirow{3}{*}{\textbf{Experiment}}} & \multicolumn{4}{c|}{\multirow{2}{*}{\textbf{With Central Region}}}                      & \multicolumn{4}{c|}{\multirow{2}{*}{\textbf{Without Central Region}}}                     \\
\multicolumn{1}{|l|}{}                  & \multicolumn{4}{c|}{}                                                                   & \multicolumn{4}{c|}{}                                                                    \\ \cline{2-9} 
\multicolumn{1}{|l|}{}                  & \textbf{Full}    & \textbf{Fixed $\Gamma$}    & \multicolumn{2}{c|}{\textbf{Fixed $M_s$}} & \textbf{Full}     & \textbf{Fixed $\Gamma$}     & \multicolumn{2}{c|}{\textbf{Fixed $M_s$}} \\ \hline
\textbf{No Crossed Terms}               & 1                & 5                       & $\sim$               & $\sim$              & 9                 & 13                       & $\sim$               & $\sim$              \\ \hline
\textbf{Fixed $a_s$}                       & 2                & 6                       & 17                   & 21                  & 10                & 14                       & 19                   & 23                  \\ \hline
\textbf{Full}                           & 3                & 7                       & 18                   & 22                  & 11                & 15                       & 20                   & 24                  \\ \hline
\textbf{No Dark Matter}                 & 4                & 8                       & $\sim$               & $\sim$              & 12                & 16                       & $\sim$               & $\sim$              \\ \hline
\end{tabular}
\caption[Characteristics of all the experiments]{Characteristics of all the experiments in our modelling. Every experiment is represented with a number for reference in the analysis section. For all the studied constraints (such as the central region, stellar mass, stellar scalength, etc), the experiments are organized as shown above.}
\end{table}

The method that we use for finding the best values of each and every parameter in our experiments had to be able to calculate several large integrals and optimize all the given parameters. For this purpose we wrote some C-based programs that contained the integrals given in equations 4.10 and 4.13 and used many \textit{gsl} routines that allow us to make integrals, interpolations and optimize the large number of calculations. Because the functions that we wanted to fit did not have an analytic solution, we decided use a $\chi^{2}$ minimization method instead. 

This approximation method allows to fit curves to the observational data so that the parameters would be optimized, that is, the combination of parameters would allow the curve to fit well the observational data. Our programs were set so that the $\chi^{2}$ was calculated for every combination of the parameters (every entry of the parameters matrix) but it will only save the values of the parameters that make the smallest $\chi^{2}$. If we call $\sigma_{M}(r_{i})$ the value of our modelling for the radial bin $i$ and if $\sigma_{i}$ is the observational value of the projected velocity dispersion in the same bin, we show that our calculation of this approximation method is given by 

\begin{equation}
\chi^{2}=\sum_{i=1}^{n}\frac{1}{N_{i}}{\left(\sigma_{M}\left(r_{i}\right)-\sigma_{i}\right)}^{2}
\end{equation}

Where $N_{i}$ is the number of velocities used to calculate the velocity dispersion in bin $i$ (this was included to give more weight to the radial bins that were calculated with a higher number of data), and $n$ is the number of radial bins.

Because we needed to vary the parameters as much as possible to take into account all the entries of the parameter matrix. We started every experiment by running the programs with a large range using the literature data and reported values as a reference, and we used a big $\Delta$ for every parameter because the computational time required for every fit is too long for a small shift in the values of the parameters. This is summarized in table 4.2.

\begin{table}[H]
\centering
\label{my-label}
\begin{tabular}{|c|c|c|c|c|c|c|}
\hline
\multicolumn{7}{|c|}{\textbf{Thick Runs}}                                                                          \\ \hline
               & \textbf{$\mathbf{\Gamma}$} & \textbf{$\mathbf{\beta}$} & \textbf{$\mathbf{M_{s}}(10^{5} M_{\odot})$} & \textbf{$\mathbf{M_{dm}}(10^{5} M_{\odot})$} & \textbf{$\mathbf{a_{s}}(pc)$} & \textbf{$\mathbf{a_{dm}}(pc)$} \\ \hline
\textbf{$\mathbf{\Delta}$}  & 0.2  & 0.2      & 5.0     & 5.0     & 5.0      & 5.0            \\ \hline
\textbf{Range} & 0.1 - 3.0      & 0.001 - 1.0        & 1.0 - 80.0      & 1.0 - 80.0   & 1.0 - 60.0      & 1.0 - 60.0            \\ \hline
\end{tabular}
\caption[Thick runs characteristics]{Characteristics of the first runs of our modelling that involve a huge range in the parameter space (to reach for all the possible values of the parameters) and a thick $\Delta$ (for computational time purposes).}
\end{table}

After these programs have been run, and we have found some values of the parameters that best fit the observational data, we do a second run of the programs for each experiments with a smaller delta around the fitted parameter in the first run to make a better approximation of the real values as shown in table 4.3.

\begin{table}[H]
\centering
\label{my-label}
\begin{tabular}{|c|c|c|c|c|c|c|}
\hline
\multicolumn{7}{|c|}{\textbf{Refined Runs}}                                                                          \\ \hline
               & \textbf{$\mathbf{\Gamma}$} & \textbf{$\mathbf{\beta}$} & \textbf{$\mathbf{M_{s}}(10^{5} M_{\odot})$} & \textbf{$\mathbf{M_{dm}}(10^{5} M_{\odot})$} & \textbf{$\mathbf{a_{s}}(pc)$} & \textbf{$\mathbf{a_{dm}}(pc)$} \\ \hline
\textbf{$\mathbf{\Delta}$}  &  0.1 &  0.01     &  0.2    &  0.2   &  0.2    &  0.2          \\ \hline
\textbf{Range} & $\Gamma_{1}\pm 0.2$   & $\beta_{1}\pm 0.03$        & $M_{s}_{1}\pm 2.0$      & $M_{dm}_{1}\pm 2.0$   & $a_{s}_{1}\pm 2.0$      & $a_{dm}_{1}\pm 2.0$            \\ \hline
\end{tabular}
\caption[Characteristics of the refined runs of the modelling]{Characteristics of the refined runs of the modelling. Note that the range used in these runs is a small region around the values found with the thick runs of the programs. Also, the $\Delta$ around which the parameters change is very small compared to the thick runs, so that the fitting of all the parameters is effectively refined.}
\end{table}

The results and interpretation of all the experiments are explained in the next section.

\section{Results}

In this section, we discuss the results given by the fitting of all the conducted experiments after the refined runs of the programs have been made. Every experiment will be referred to it's given number in table 4.1, where it's characteristics are explicitly shown.

\subsection{Full fits with inner region (Experiments 1,2,3,4)}

This set of experiments consist of the variation of all the parameters with the velocity dispersion profile that includes the central region of the globular cluster (the first two radial bins). The results of the fitted curve with the optimized parameters are shown in figure 4.8

\begin{figure}[H]
\centering
\includegraphics[width=15cm]{images/all_params_refinado_12.png}
\caption[Best fit of the full model with the inner region]{Best fit to the observational data of experiments 1,2,3, and 4 for our full model with the inner region of the cluster (in orange) and the observational data of radial velocity dispersions in the chosen radial bins (red). They are shown in different plots because in some cases they overlap too much for their behaviour to be clearly seen.}

\end{figure}

Note that in the above figures, the fitted curves reproduced the observational data consistently although every line has its own characteristics. In the case of experiments 1,3 and 4, there is an awkward behaviour of the line in the inner region that didn't seem to be a problem in the solution of the large integrals in equation 4.13 so it could be due to physical reasons such as a massive body in that region (we don't have enough arguments to explain why this strange behaviour only occurs in the fits where we include the inner region of the cluster). 

The fitted values for the parameters that reproduce figure 4.8 are shown in table 4.4.

\begin{table}[H]
\begin{center}
\begin{tabular}{| c | c | c | c | c | c | c| }
    \hline
    \textbf{Experiment} & $\mathbf{\beta}$ & $\mathbf{a_{dm}} (pc)$ & $\mathbf{a_{s}} (pc)$ & $\mathbf{M_{dm}}$ ($M_{\odot}$) & $\mathbf{M_{s}}$ ($M_{\odot}$) & $\mathbf{\Gamma}$\\ \hline
	No Crossed terms (1) & $0.62$ &	$15.8$ &	$29.0$ &	$5.6 \times 10^{5}$ &	$8.0 \times 10^{4}$ &	$2.2$\\ \hline
	Fixed $a_s$ (2) &	$0.0001$ &	$9.0$ &	$2.23$ &	$1.12 \times 10^{5}$ &	$1.06 \times 10 ^{6}$ &	$1.5$\\ \hline
	Full (3) &	$0.46$ &	$15.2$ &	$52.6$ &	$9 \times 10^{5}$ &	$9 \times 10^{5}$ &	$0.38$\\ \hline
	No Dark Matter (4) &	$0.26$ & $\thicksim$	& $3.64$  & $\thicksim$ & $  1.98 \times 10^{6}$ & 	$1.24$\\
    \hline
  \end{tabular} 
\caption[Optimized parameters for our full model with the inner region.]{The optimized parameters for each of the experiments for our full model with the inner region of the cluster. }
\end{center}
\end{table}

\subsection{Full fits without the inner region (Experiments 9,10,11,12)}

In this set of experiments we do the variation of all the parameters and we use only the outer region of the cluster (2.0 to 45 arcmin) because we want to avoid the problems in the center of the cluster and search for a smoother behaviour in the fitting of our model. 

Figure 4.9 shows the fitted lines for experiments 9,10,11 and 12.

\begin{figure}[H]
\centering
\includegraphics[width=15cm]{images/all_params_refinado_10.png}
\caption[Best fit of the full model without the inner region]{Best fit to the observational data of the four experiments for our full model without the inner region of the cluster.}
\end{figure}

The first thing we notice is that the strange behaviour in the centre of the cluster is different than in the previous fits, and is only shown in experiments 9 and 11. The curves are very satisfactory because they fit very well the observational data and show a smooth behaviour as expected. 

The optimized parameters for this set of experiments are displayed in table 4.5.

\begin{table}[H]
\begin{center}
\begin{tabular}{| c| c| c| c| c| c| c|}
    \hline
    \textbf{Experiment} & $\mathbf{\beta}$ & $\mathbf{a_{dm}} (pc)$ & $\mathbf{a_{s}} (pc)$ & $\mathbf{M_{dm}}$ ($M_{\odot}$) & $\mathbf{M_{s}}$ ($M_{\odot}$) & $\mathbf{\Gamma}$\\ \hline
	No Crossed terms (9) & $0.6$ &	$16$ &	$52.8$ &	$2.1 \times 10^{6}$ &	$2.72 \times 10^{6}$ &	$2.3$\\ \hline
	Fixed (10) $a_s$ &	$0.79$ &	$57.9$ &	$2.23$ &	$8 \times 10^{5}$ &	$3.43 \times 10 ^{6}$ &	$2.1$\\ \hline
	Full (11) &	$0.04$ &	$11.8$ &	$57.8$ &	$6 \times 10^{5}$ &	$9 \times 10^{5}$ &	$2.1$\\ \hline
	No Dark Matter (12) &	$0.78$ &	$\thicksim$ & $2.96$ &	$\thicksim$ & $ 3 \times 10^{6}$ & 	$1.94$\\
    \hline
  \end{tabular} 
\caption[Optimized parameters for our full model without the inner region.]{The optimized parameters for each of the experiments for our full model without the inner region of the cluster.}
\end{center}
\end{table}

\subsection{Fixed $\Gamma$ ratio with the inner region (Experiments 5,6,7,8)}

In the following set of experiments, we constrained the mass-to-light ratio to that found by Van de Ven et al. 2005 with a value of $\Gamma = 2.5$. 

The first group of experiments are 5, 6, 7 and 8 as named in table 4.1. The constant mass-to-light ratio $\Gamma$ allows us to make a slightly more precise fitting of the other parameters because it needs a shorter computational time to calculate the other parameters. The fitted curves are shown in figure 4.10. 

\begin{figure}[H]
\centering
\includegraphics[width=15cm]{images/fix_gamma_refinado_12.png}
\caption[Best fit of our model with a fix mass-to-light ratio with the inner region]{Best fit to the observational data of the four experiments for our model with a fix mass-to-light ratio with the inner region of the cluster.}
\end{figure}

table 4.6.

\begin{table}[H]
\begin{center}
\begin{tabular}{| c| c| c| c| c| c| c|}
    \hline
    \textbf{Experiment} & $\mathbf{\beta}$ & $\mathbf{a_{dm}} (pc)$ & $\mathbf{a_{s}} (pc)$ & $\mathbf{M_{dm}}$ ($M_{\odot}$) & $\mathbf{M_{s}}$ ($M_{\odot}$) & $\mathbf{\Gamma}$\\ \hline
	No Crossed terms (5) & $0.35$ &	$16.4$ &	$55.62$ &	$1.62 \times 10^{6}$ &	$2.1 \times 10^{6}$ &	$2.5$\\ \hline
	Fixed $a_s$ (6) &	$0.001$ &	$3.0$ &	$2.23$ &	$3 \times 10^{5}$ &	$5.0 \times 10 ^{5}$ &	$2.5$\\ \hline
	Full &	$0.72$ (7) &	$20.0$ &	$44.4$ &	$5.2 \times 10^{5}$ &	$8.0 \times 10^{4}$ &	$2.5$\\ \hline
	No Dark Matter (8) &	$0.001$ &	$\thicksim$ & $ 3.15$ &	$\thicksim$ & $ 1.52 \times 10^{6}$ & 	$2.5$\\
    \hline
  \end{tabular} 
\caption[Optimized parameters for our fix mass-to-light ratio model with the inner region.]{The optimized parameters for each of the experiments for our fix mass-to-light ratio model with the inner region of the cluster.}
\end{center}
\end{table}

\subsection{Fixed $\Gamma$ ratio without the inner region (Experiments 13,14,15,16)}

Again, we set the mass-to-light ratio to $2.5$ and vary the rest of the parameters exluding the inner region of the cluster (0.5-2.0 arcmin) in experiments 13, 14, 15 and 16. The fitted curves are shown in figure 4.11

\begin{figure}[H]
\centering
\includegraphics[width=15cm]{images/fix_gamma_refinado_10.png}
\caption[Best fit of our model with a fix mass-to-light ratio without the inner region]{Best fit to the observational data of the four experiments for our model with a fix mass-to-light ratio without the inner region f the cluster.}
\end{figure}

 and table 4.7.

\begin{table}[H]
\begin{center}
\begin{tabular}{| c| c| c| c| c| c| c|}
    \hline
    \textbf{Experiment} & $\mathbf{\beta}$ & $\mathbf{a_{dm}} (pc)$ & $\mathbf{a_{s}} (pc)$ & $\mathbf{M_{dm}}$ ($M_{\odot}$) & $\mathbf{M_{s}}$ ($M_{\odot}$) & $\mathbf{\Gamma}$\\ \hline
	No Crossed terms (13) & $0.2$ &	$15.2$ &	$59.8$ &	$1.4 \times 10^{6}$ &	$2.1 \times 10^{6}$ &	$2.5$\\ \hline
	Fixed $a_s$ (14) &	$0.801$ &	$3.0$ &	$2.23$ &	$5 \times 10^{5}$ &	$1.0 \times 10 ^{5}$ &	$2.5$\\ \hline
	Full (15) &	$0.9$ &	$16.0$ &	$44.6$ &	$1.62 \times 10^{6}$ &	$1.42 \times 10^{6}$ &	$2.5$\\ \hline
	No Dark Matter (16) &	$0.38$ &	$\thicksim$ & $ 2.38$ &	$\thicksim$ & $  2.03 \times 10^{6}$ & 	$2.5$\\
    \hline
  \end{tabular} 
\caption[Optimized parameters for our fix mass-to-light ratio model without the inner region.]{The optimized parameters for each of the experiments for our fix mass-to-light ratio model without the inner region of the cluster.}
\end{center}
\end{table} 

Now, we set the constraint of a stellar mass to the mass that we found with Starlight and get the following results.

\subsection{Fixed Stellar Mass with the inner region (Experiments 17,18)}

We noted that the experiments where we didn't use the crossed terms in equation 4.13 differed very strongly from the experiments where we solved the whole equation. This means that those crossed terms need to be included in the correct fitting so we decided not to waste time running the ``No Crossed Terms" programs in the following experiments. Also, the ``No Dark Matter" experiments are not conducted since we would already be already constraining the total mass to be the stellar mass found with Starlight so that it would be a waste of time to run those programs.

The experiments we run are ``Full" and ``Fix $a_s$" named as 17 and 18 in table 4.1 and the results are shown in figure 4.12.

\begin{figure}[H]
\centering(14)
\includegraphics[width=15cm]{images/Starlight_2.png}
\caption[Best fit of our model with the mass found with the Starlight procedures with the inner region]{Best fit to the observational data of the four experiments for our model with the Starlight procedures with the inner region of the cluster. Note that the fitting with a fixed stellar scalength was very bad and didn't give us any valuable results of the parameters.}
\end{figure}

and table 4.8 

\begin{table}[H]
\begin{center}
\begin{tabular}{| c| c| c| c| c| c| c|}
    \hline
    \textbf{Experiment} & $\mathbf{\beta}$ & $\mathbf{a_{dm}} (pc)$ & $\mathbf{a_{s}} (pc)$ & $\mathbf{M_{dm}}$ ($M_{\odot}$) & $\mathbf{M_{s}}$ ($M_{\odot}$) & $\mathbf{\Gamma}$\\ \hline
	Fixed $a_s$ (17) &	$0.88$ &	$56.98$ &	$2.23$ &	$8 \times 10^{4}$ &	$6.61 \times 10 ^{6}$ &	$1.6$\\ \hline
	Full (18) &	$0.96$ &	$7.6$ &	$12.0$ &	$6.88 \times 10^{5}$ &	$6.61 \times 10^{6}$ &	$0.9$\\ \hline
  \end{tabular} 
\caption[Optimized parameters for our model with the mass found with the Starlight procedures with the inner region.]{The optimized parameters for each of the experiments for our model with the mass found with the Starlight procedures with the inner region of the cluster.}
\end{center}
\end{table}

\subsection{Fixed Stellar Mass without the inner region (Experiments 19,20)}

Once again, we want to see how the centre of the cluster affects our results so we run experiments 19 and 20 for the outer region of $\omega$ Centauri. The fitted cures are shown in figure 4.13. 

\begin{figure}[H]
\centering
\includegraphics[width=15cm]{images/Starlight_1.png}
\caption[Best fit of our model with the mass found with the Starlight procedures without the inner region]{Best fit to the observational data of the four experiments for our model without the Starlight procedures with the inner region of the cluster.}
\end{figure}

and table 4.9

\begin{table}[H]
\begin{center}
\begin{tabular}{| c| c| c| c| c| c| c|}
    \hline
    \textbf{Experiment} & $\mathbf{\beta}$ & $\mathbf{a_{dm}} (pc)$ & $\mathbf{a_{s}} (pc)$ & $\mathbf{M_{dm}}$ ($M_{\odot}$) & $\mathbf{M_{s}}$ ($M_{\odot}$) & $\mathbf{\Gamma}$\\ \hline
	Fixed $a_s$ (19) &	$0.95$ &	$58.0$ &	$2.23$ &	$8 \times 10^{4}$ &	$6.61 \times 10 ^{6}$ &	$2.3$\\ \hline
	Full (20) &	$0.96$ &	$7.36$ &	$50.0$ &	$1.35 \times 10^{6}$ &	$6.61 \times 10^{6}$ &	$1.88$\\ \hline
  \end{tabular} 
\caption[Optimized parameters for our model with the mass found with the Starlight procedures without the inner region.]{The optimized parameters for each of the experiments for our model with the mass found with the Starlight procedures without the inner region of the cluster.}
\end{center}
\end{table}


As the two last experiments, we want to take into account the big error bar in our calculation of the stellar mass found by Starlight, so comparing it with the reported values we decided to use the lowest possible value that an estimated error bar of 25\% would allow us to use. In this case the value of the stellar mass is $M_{s}=4.9 \times 10^{6} M_{\odot}$

\subsection{Final Fixed Stellar Mass with the inner region (Experiments 21,22)}

shown in figure 4.14

\begin{figure}[H]
\centering
\includegraphics[width=15cm]{images/Starlight_25_12.png}
\caption[Best fits for our model with a mass value based on the Starlight procedures with the inner region.]{Best fit for each of the experiments for our model with a mass value found with the Starlight procedures with the inner region of the cluster.}
\end{figure}

 and table 4.10.

\begin{table}[H]
\begin{center}
\begin{tabular}{| c| c| c| c| c| c| c|}
    \hline
    \textbf{Experiment} & $\mathbf{\beta}$ & $\mathbf{a_{dm}} (pc)$ & $\mathbf{a_{s}} (pc)$ & $\mathbf{M_{dm}}$ ($M_{\odot}$) & $\mathbf{M_{s}}$ ($M_{\odot}$) & $\mathbf{\Gamma}$\\ \hline
	Fixed $a_s$ (21) &	$0.88$ &	$56.98$ &	$2.23$ &	$9 \times 10^{4}$ &	$4.9 \times 10 ^{6}$ &	$0.82$\\ \hline
	Full (22) &	$0.98$ &	$13.26$ &	$31.98$ &	$1.98 \times 10^{6}$ &	$4.9 \times 10^{6}$ &	$0.66$\\ \hline
  \end{tabular} 
\caption[Optimized parameters for our model with a mass value based on the Starlight procedures with the inner region.]{The optimized parameters for each of the experiments for our model with a mass value found with the Starlight procedures with the inner region of the cluster.}
\end{center}
\end{table}

\subsection{Final Fixed Stellar Mass without the inner region (Experiments 23,24)}

And finally, excluding the innermost region of the cluster, we have the results of the best fits shown in figure 4.15

\begin{figure}[H]
\centering
\includegraphics[width=15cm]{images/Starlight_25_10.png}
\caption[Best fits for our model with a mass value based on the Starlight procedures without the inner region.]{Best fit for each of the experiments for our model with a mass value found with the Starlight procedures without the inner region of the cluster. Note that again, the fitting for the model with the fixed stellar scalength was very bad and didn't give any trustworthy results.}
\end{figure}

 and table 4.11

\begin{table}[H]
\begin{center}
\begin{tabular}{| c| c| c| c| c| c| c|}
    \hline
    \textbf{Experiment} & $\mathbf{\beta}$ & $\mathbf{a_{dm}} (pc)$ & $\mathbf{a_{s}} (pc)$ & $\mathbf{M_{dm}}$ ($M_{\odot}$) & $\mathbf{M_{s}}$ ($M_{\odot}$) & $\mathbf{\Gamma}$\\ \hline
	Fixed $a_s$ (23) &	$0.88$ &	$56.98$ &	$2.23$ &	$9 \times 10^{4}$ &	$4.9 \times 10 ^{6}$ &	$1.1$\\ \hline
	Full (24) &	$0.92$ &	$7.26$ &	$37.98$ &	$1.35 \times 10^{6}$ &	$4.9 \times 10^{6}$ &	$0.72$\\ \hline
  \end{tabular} 
\caption[Optimized parameters for our model with a mass value based on the Starlight procedures without the inner region.]{The optimized parameters for each of the experiments for our model with a mass value found with the Starlight procedures without the inner region of the cluster.}
\end{center}  
\end{table}

The conclusions of the modelling is discussed in the following chapter.
\chapter{Modelling}

We used various techniques for the mass modelling of $\omega$ Centauri, so that we could make a good approximation to its dynamic and stellar mass. In order to do this, we decided to use the Hernquist profile and made some modifications on it so that dark matter contributions would also be included.  

\section{Modified Hernquist Model}

Our modelling is based on significant modifications to the Hernquist profile (Hernquist 1989).  We use this model because it is a well known model that closely approximates the de Vaucouleurs law for elliptical galaxies and has analytical solutions that can be useful for our computational purposes. As introduced in chapter 2, the density profile for this model is

\begin{equation}
\rho(r)=\frac{M}{2\pi}\frac{a}{r}\frac{1}{\left(r+a\right)^{3}}
\end{equation}

It's associated cumulative mass is

\begin{equation}
M(r)=M\frac{r^{2}}{(r+a)^{2}}
\end{equation}
 
And the surface brightness 
 
 \begin{equation}
 I(R)=\frac{M}{2\pi a^{2}\Gamma\left(1-s^{2}\right)^{2}}\left[\left(2+s^{2}\right)X(s)-3\right]
 \end{equation}
 
where $s=R/a$, $R$ is the projected radius and:

\begin{equation}
X(s)=\frac{1}{\sqrt{1-s^{2}}}sech^{-1}s\qquad for\qquad0\leq s\leq1
\end{equation}

\begin{equation}
X(s)=\frac{1}{\sqrt{s^{2}-1}}sec^{-1}s\qquad for\qquad1\leq s<\infty
\end{equation}

For computational simplicity we have written some of the trigonometric functions just like Hernquist did: $sec^{-1}s = cos^{-1}(1/s)$ and $sech^{-1} = ln[(1+\sqrt{1-s^{2}})/s]$. As mentioned in the theoretical framework chapter, the line of sight velocity dispersion in the more general case with the anisotropy parameter different from 0 we have

\begin{equation}
I(R)\sigma_{p}^{2}(R)=\frac{2}{\Gamma}\int_{R}^{\infty}\left(1-\beta\frac{R^{2}}{r^{2}}\right)\frac{\rho\bar{v_{r}^{2}}rdr}{\sqrt{r^{2}-R^{2}}}
\end{equation}

And the radial velocity dispersion (introduced in chapter 2), in terms of the potential and the density is

\begin{equation}
\bar{v_{r}^{2}}=\sigma_{r}^{2}=\frac{1}{\rho(r)}\int_{r}^{\infty}\rho(r)\frac{d\phi}{dr}dr
\end{equation}

Where 

\begin{equation}
\frac{d\phi}{dr}=\frac{GM(r)}{r^{2}}
\end{equation}

So the projected velocity dispersion becomes:

\begin{equation}
\sigma_{p}^{2}(R)=\frac{2}{I(R)\Gamma}\int_{R}^{\infty}\left(1-\beta\frac{R^{2}}{r^{2}}\right)\left(G\int_{r}^{\infty}\frac{\rho(r)M(r)}{r^{2}}dr\right)\frac{rdr}{\sqrt{r^{2}-R^{2}}}
\end{equation}

We do various experiments for our modelling, the simplest of all models is the one where we assume that the cluster doesn't have dark matter whatsoever, in this case there only will be one mass contribution and only one scalength (the stellar scalength) so the solution of the last equation would be:

\begin{equation}
\sigma_{p}^{2}(R)=\frac{GM^{2}a}{I(R)\Gamma\pi}\int_{R}^{\infty}\alpha(r)\left(\frac{\log{\left(\frac{a+r}{r}\right)}}{a^{5}}-\frac{25a^{3}+52a^{2}r+42ar^{2}+12r^{3}}{12a^{4}\left(a+r\right)^{4}}\right)dr
\end{equation}

Where $a$ is the scalength and where, in order to shorten the equation we take $\alpha(r)$ as 

\begin{equation}
\alpha(r)=\left(1-\beta\frac{R^{2}}{r^{2}}\right)\frac{r}{\sqrt{r^{2}-R^{2}}}
\end{equation}

Now, as we want to focus on the dark matter content of the cluster, we assume that the mass of the cluster is the sum of the mass of stars and the mass of non-baryonic matter so that their contributions to the density and mass profiles become:

\begin{equation}
\rho(r)=\rho_{s}(r)+\rho_{dm}(r)\qquad and \qquad M(r)=M_{s}(r)+M_{dm}(r)
\end{equation} 

In this case, the projected velocity dispersion takes a much more complicated form as follows:

\begin{equation}
\begin{aligned}	
\sigma_{p}^{2}(R) &= \frac{G}{I(R)\Gamma\pi}\int_{R}^{\infty}\alpha(r)\Biggl[\underbrace{\int_{r}^{\infty}\frac{M_{s}^{2}a_{s}dr}{r\left(r+a_{s}\right)^{5}}}_{\mathbf{A}(r)}+\underbrace{\int_{r}^{\infty}\frac{M_{s}M_{dm}a_{s}dr}{r\left(r+a_{s}\right)^{3}\left(r+a_{dm}\right)^{2}}}_{\mathbf{B}(r)}\right\\     &+ \underbrace{\int_{r}^{\infty}\frac{M_{dm}M_{s}a_{dm}dr}{r\left(r+a_{dm}\right)^{3}\left(r+a_{s}\right)^{2}}}_{\mathbf{C}(r)}+\underbrace{\int_{r}^{\infty}\frac{M_{dm}^{2}a_{dm}dr}{r\left(r+a_{dm}\right)^{5}}}_{\mathbf{D}(r)}\Biggr] dr
\end{aligned}
\end{equation}

The functional form of the density involves the use of a stellar scalength ($a_{s}$) and a dark matter scalength ($a_{dm}$). Now, the integrals $\mathbf{A}(r),\mathbf{B}(r),\mathbf{C}(r)$ and $\mathbf{D}(r)$ have the following analytical solutions:

\begin{equation}
\textbf{A}(r)=a_{s}\left(-\frac{25a_{s}^{3}+52a_{s}^{2}r+42a_{s}r^{2}+12r^{3}}{12a_{s}^{4}\left(a_{s}+r\right)^{4}}+\frac{\log{\left[\frac{a_{s}+r}{r}\right]}}{a_{s}^{5}}\right)
\end{equation}

\begin{equation}
\textbf{D}(r)=a_{dm}\left(-\frac{25a_{dm}^{3}+52a_{dm}^{2}r+42a_{dm}r^{2}+12r^{3}}{12a_{dm}^{4}\left(a_{dm}+r\right)^{4}}+\frac{\log{\left[\frac{a_{dm}+r}{r}\right]}}{a_{dm}^{5}}\right)
\end{equation}

\begin{equation}
\textbf{B}(r)=\frac{\left(M_{s}M_{dm}\right)\left(\mathbf{b_{2}}+\mathbf{b_{3}}+a_{s}\left(-\left(a_{s}-a_{dm}\right)a_{dm}\mathbf{b_{4}}+\mathbf{b_{5}}\right)\right)}{\mathbf{b_{1}}}
\end{equation}

\begin{equation}
With \left\lbrace
\begin{array}{lllll}
\mathbf{b_{1}}=2a_{s}^{2}(a_{s}-a_{dm})^{4}a_{dm}^{2}(a_{s}+r)^{2}(a_{dm}+r)\\
\mathbf{b_{2}}=-2\left(a_{s}-a_{dm}\right)^{4}\left(a_{s}+r\right)^{2}\left(a_{dm}+r\right)\log{r}\\
\mathbf{b_{3}}=2a_{dm}^{2}\left(6a_{s}^{2}-4a_{s}a_{dm}+a_{dm}^{2}\right)
\left(a_{s}+r\right)^{2}\left(a_{dm}+r\right)\log{[a_{s}+r]}\\
\begin{aligned}	
\mathbf{b_{4}} &= 2a_{s}^{4}+4a_{s}^{3}r-2a_{dm}r(a_{dm}+r)+3a_{s}a_{dm}\left(-a_{dm}^{2}+a_{dm}r+2r^{2}\right)\\      &+a_{s}^{2}\left(7a_{dm}^{2}+7a_{dm}r+2r^{2}\right)
\end{aligned}\\
\mathbf{b_{5}}=2a_{s}^{2}\left(a_{s}-4a_{dm}\right)\left(a_{s}+r\right)^{2}\left(a_{dm}+r\right)\log{[a_{dm}+r]}
\end{array}
\right.
\end{equation} 

\begin{equation}
\textbf{C}(r)=\frac{\left(M_{dm}M_{s}\right)\left(\mathbf{c_{2}}+\mathbf{c_{3}}+a_{dm}\left(-\left(a_{dm}-a_{s}\right)a_{s}\mathbf{c_{4}}+\mathbf{c_{5}}\right)\right)}{\mathbf{c_{1}}}
\end{equation}

\begin{equation}
With \left\lbrace
\begin{array}{lllll}
\mathbf{c_{1}}=2a_{dm}^{2}(a_{dm}-a_{s})^{4}a_{s}^{2}(a_{dm}+r)^{2}(a_{s}+r)\\
\mathbf{c_{2}}=-2\left(a_{dm}-a_{s}\right)^{4}\left(a_{dm}+r\right)^{2}\left(a_{s}+r\right)\log{r}\\
\mathbf{c_{3}}=2a_{s}^{2}\left(6a_{dm}^{2}-4a_{dm}a_{s}+a_{s}^{2}\right)\left(a_{dm}+r\right)^{2}
\left(a_{s}+r\right)\log{[a_{dm}+r]}\\
\begin{aligned}	
\mathbf{c_{4}} &= 2a_{dm}^{4}+4a_{dm}^{3}r-2a_{s}r(a_{s}+r)+3a_{dm}a_{s}\left(-a_{s}^{2}+a_{s}r+2r^{2}\right)\\      &+a_{dm}^{2}\left(7a_{s}^{2}+7a_{s}r+2r^{2}\right)
\end{aligned}\\
\mathbf{c_{5}} =2a_{dm}^{2}\left(a_{dm}-4a_{s}\right)\left(a_{dm}+r\right)^{2}\left(a_{s}+r\right)\log{[a_{s}+r]}
\end{array}
\right.
\end{equation} 

As we mentioned in the last chapter, we used many databases for projected radial velocities in order to properly do our fit and modelling. The data is shown in figure 4.1

\begin{figure}[H]
\centering
\includegraphics[width=12cm]{images/vel_vs_rad.png}
\caption[Pg]{Pd}
\end{figure}

Because we need to calculate the projected velocity dispersion profile, we need to cut in radial bins as shown in figure 4.2 

\begin{figure}[H]
\centering
\includegraphics[width=12cm]{images/vel_vs_rad_bins.png}
\caption[Pg]{Pd}
\end{figure}

The velocity dispersion is the standard deviation of the velocity

\begin{equation}
\sigma = f(v) = \sqrt{\frac{\sum_{i=1}^{n}\left(v_{i}-v\right)^{2}}{N}}
\end{equation}

In order to do a proper modelling and fitting we need to take into account the error associated with $\sigma_{p}$. Because all the databases provided the associated error to the measurements of the radial velocities, we made the proper calculation of the error of the velocity dispersion using the general error propagation formula

\begin{equation}
\sigma(v\pm\Delta v)\approx \sigma(v)\pm \underbrace{\frac{\partial \sigma}{\partial v}\Delta v}_{\Delta \sigma}
\end{equation}

\begin{equation}
\Delta \sigma = \frac{1}{\sqrt{N}}\left(\sum_{i=1}^{n}\left(v_{i}-v\right)^{2}\right)^{-1/2}\sum_{i=1}^{n}\left(v_{i}-v\right)\Delta v
\end{equation}

With the radial projected velocity dispersion (and its error) we can start with the fitting to find the optimized parameters in our various experiments. Because the functions that we wanted to fit did not have an analytic solution, we decided use a $\chi^{2}$ method instead. 

This approximation method allows to fit curves to the observational data so that the parameters would be optimized, that is, the combination of parameters would allow the curve to fit well the observational data. Our programs written in C were set so that the $\chi^{2}$ was calculated for every combination of the parameters (every entry of the parameters matrix) but it will only save the values of the parameters that make the smallest $\chi^{2}$. If we call $\sigma_{M}(r_{i})$ the value of our modelling for the radial bin $i$ and if $\sigma_{i}$ is the observational value of the projected velocity dispersion in the same bin, we show that our calculation of this approximation method is given by 

\begin{equation}
\chi^{2}=\sum_{i=1}^{n}\frac{1}{N_{i}}{\left(\sigma_{M}\left(r_{i}\right)-\sigma_{i}\right)}^{2}
\end{equation}

Where $N_{i}$ is the number of velocities used to calculate the velocity dispersion in bin $i$ (this was included to give more weight to the radial bins that were calculated with a higher number of data), and $n$ is the number of radial bins. By doing these runs over big $\Delta$s and then refining them, we obtain the following results for each of our experiments.
  
\subsection{Full Modelling}

This first set of experiments consisted of the variation of every parameter of our modelling, that is, dark matter mass $M_{dm}$, stellar mass $M_{s}$, dark matter scalength $a_{dm}$, stellar scalength $a_{s}$, mass-to-light ratio $\Gamma$ and parameter of anisotropy $\beta$.

Because some authors (Jalali et al. 2012) state that there might be a medium-mass black hole at the center of $\omega$ Centauri, we decided to conduct two different fittings, one with the central region of the cluster (12 radial bins) and other without the central region of the cluster (10 radial bins), and in each group of experiments we made four basic assumptions to take into account all the possibilities of our modelling:

\textbf{\textit{i)}} We don't use the crossed terms in equation 4.13, because we wanted to see how relevant they were in the modelling, we call this ``\textit{No crossed terms}". 

\textbf{\textit{ii)}} We assume a constant stellar scalength taken from the literature, we call this ``\textit{Fix $a_{s}$}".

\textbf{\textit{iii)}} We don't assume any of our parameters to be constant, we call this ``\textit{full modelling}". 

\textbf{\textit{iv)}} We assume the cluster doesn't have any dark matter so the solution of the equation 4.9 is much simpler and it's given by 4.10, we call this ``\textit{No Dark matter}".

Now, let's see the results for all of these experiments with the graphs that show the fitting and the optimized values of the parameters found for the minimum $\chi^{2}$

\textbf{Full fits with 12 radial bins}

We figure 4.3

\begin{figure}[H]
\centering
\includegraphics[width=15cm]{images/all_params_refinado_12.png}
\caption[Pg]{Pd}
\end{figure}

\begin{table}[H]
\begin{center}
\begin{tabular}{| c | c | c | c | c | c | c| }
    \hline
    \textbf{Experiment} & $\mathbf{\beta}$ & $\mathbf{a_{dm}} (pc)$ & $\mathbf{a_{s}} (pc)$ & $\mathbf{M_{dm}}$ ($M_{\odot}$) & $\mathbf{M_{s}}$ ($M_{\odot}$) & $\mathbf{\Gamma}$\\ \hline
	No Crossed terms & $0.62$ &	$15.8$ &	$29.0$ &	$5.6 \times 10^{5}$ &	$8.0 \times 10^{4}$ &	$2.2$\\ \hline
	Fix $a_s$ &	$0.0001$ &	$9.0$ &	$2.23$ &	$1.12 \times 10^{5}$ &	$1.06 \times 10 ^{6}$ &	$1.5$\\ \hline
	Full &	$0.46$ &	$15.2$ &	$52.6$ &	$9 \times 10^{5}$ &	$9 \times 10^{5}$ &	$0.38$\\ \hline
	No Dark Matter &	$0.26$ & -------	& $ a = 3.64$  &	------- & $  M = 1.98 \times 10^{6}$ & 	$1.24$\\
    \hline
  \end{tabular} 
\caption[It]{Ie}
\end{center}
  
\end{table}

\textbf{Full fits with 10 radial bins}

figure 4.4

\begin{figure}[H]
\centering
\includegraphics[width=15cm]{images/all_params_refinado_10.png}
\caption[Pg]{Pd}
\end{figure}

\begin{table}[H]
\begin{center}
\begin{tabular}{| c| c| c| c| c| c| c|}
    \hline
    \textbf{Experiment} & $\mathbf{\beta}$ & $\mathbf{a_{dm}} (pc)$ & $\mathbf{a_{s}} (pc)$ & $\mathbf{M_{dm}}$ ($M_{\odot}$) & $\mathbf{M_{s}}$ ($M_{\odot}$) & $\mathbf{\Gamma}$\\ \hline
	No Crossed terms & $0.6$ &	$16$ &	$52.8$ &	$2.1 \times 10^{6}$ &	$2.72 \times 10^{6}$ &	$2.3$\\ \hline
	Fix $a_s$ &	$0.79$ &	$57.9$ &	$2.23$ &	$8 \times 10^{5}$ &	$3.43 \times 10 ^{6}$ &	$2.1$\\ \hline
	Full &	$0.04$ &	$11.8$ &	$57.8$ &	$6 \times 10^{5}$ &	$9 \times 10^{5}$ &	$2.1$\\ \hline
	No Dark Matter &	$0.78$ &	------ & $ a = 2.96$ &	------- & $  M = 3 \times 10^{6}$ & 	$1.94$\\
    \hline
  \end{tabular} 
\caption[It]{Ie}
\end{center}
  
\end{table}

\subsection{Fix mass-to-light ratio}

In this set of experiments, the mass-to-light ratio $\Gamma$ is a fix value that we found by fitting the effective radius in a de Vaucouleurs profile on the surface brightness of Eva Noyola's observational data (Noyola et al. 2013) shown in figure 4.5

\begin{figure}[H]
\centering
\includegraphics[width=10cm]{images/noyola.png}
\caption[Pg]{Surface brightness profile for $\omega$ Centauri. The circles show
Noyola's measured photometric points. The triangles show photometric points obtained
from ground based images by Trager et al. The dashed line
is Trager’s Chebychev fit. The solid line is Noyola's smooth fit that we don't use for our fitting of the scalength. Figure taken from Noyola et al. 2013}
\end{figure}

Our fitting (easily made with Mathematica) gives us an effective radius of $R_{e}=4.048$ and using the relation $R_{e}\approx1.8153a$ (Hernquist 1989), we find a stellar scalength of $a=2.23$ that we use for our set of experiments. The other parameters ($M_{dm}$, $M_{s}$, $a_{dm}$, $a_{s}$, $\beta$) were again varied to find the smallest $\chi^{2}$.

Once again, we do our experiments with 10 and 12 radial bins (for the problems that the black hole could do in the fitting).

\textbf{Fix mass-to-light ratio with 12 radial bins}

figure 4.6

\begin{figure}[H]
\centering
\includegraphics[width=15cm]{images/fix_gamma_refinado_12.png}
\caption[Pg]{Pd}
\end{figure}

\begin{table}[H]
\begin{center}
\begin{tabular}{| c| c| c| c| c| c| c|}
    \hline
    \textbf{Experiment} & $\mathbf{\beta}$ & $\mathbf{a_{dm}} (pc)$ & $\mathbf{a_{s}} (pc)$ & $\mathbf{M_{dm}}$ ($M_{\odot}$) & $\mathbf{M_{s}}$ ($M_{\odot}$) & $\mathbf{\Gamma}$\\ \hline
	No Crossed terms & $0.35$ &	$16.4$ &	$55.62$ &	$1.62 \times 10^{6}$ &	$2.1 \times 10^{6}$ &	$2.5$\\ \hline
	Fix $a_s$ &	$0.001$ &	$3.0$ &	$2.23$ &	$3 \times 10^{5}$ &	$5.0 \times 10 ^{5}$ &	$2.5$\\ \hline
	Full &	$0.72$ &	$20.0$ &	$44.4$ &	$5.2 \times 10^{5}$ &	$8.0 \times 10^{4}$ &	$2.5$\\ \hline
	No Dark Matter &	$0.001$ &	----- & $ a = 3.15$ &	----- & $  M = 1.52 \times 10^{6}$ & 	$2.5$\\
    \hline
  \end{tabular} 
\caption[It]{Ie}
\end{center}
  
\end{table}

\textbf{Fix mass-to-light ratio with 10 radial bins}

figure 4.7

\begin{figure}[H]
\centering
\includegraphics[width=15cm]{images/fix_gamma_refinado_10.png}
\caption[Pg]{Pd}
\end{figure}

\begin{table}[H]
\begin{center}
\begin{tabular}{| c| c| c| c| c| c| c|}
    \hline
    \textbf{Experiment} & $\mathbf{\beta}$ & $\mathbf{a_{dm}} (pc)$ & $\mathbf{a_{s}} (pc)$ & $\mathbf{M_{dm}}$ ($M_{\odot}$) & $\mathbf{M_{s}}$ ($M_{\odot}$) & $\mathbf{\Gamma}$\\ \hline
	No Crossed terms & $0.2$ &	$15.2$ &	$59.8$ &	$1.4 \times 10^{6}$ &	$2.1 \times 10^{6}$ &	$2.5$\\ \hline
	Fix $a_s$ &	$0.801$ &	$3.0$ &	$2.23$ &	$5 \times 10^{5}$ &	$1.0 \times 10 ^{5}$ &	$2.5$\\ \hline
	Full &	$0.9$ &	$16.0$ &	$44.6$ &	$1.62 \times 10^{6}$ &	$1.42 \times 10^{6}$ &	$2.5$\\ \hline
	No Dark Matter &	$0.38$ &	------ & $ a = 2.38$ &	------ & $  M = 2.03 \times 10^{6}$ & 	$2.5$\\
    \hline
  \end{tabular} 
\caption[It]{Ie}
\end{center}
  
\end{table}


\section{Stellar Population Synthesis with Starlight}

The stellar mass content of Globular Clusters and Galaxies can be studied through the determination of the stellar populations inside those systems since we have clear knowledge about their photometric properties. If we have information about the amount of stars of a given type inside a stellar system, we can infer how much of the system's mass is given by these populations of stars. 

The determination of the stellar populations can be done using STARLIGHT, which is a Fortran-based program that fits an observed integrated spectrum (Omega Centauri in our case) with a model spectrum which is the sum of $N_{*}$ spectral components from a pre-defined and pre-processed set of base spectra. The program does as many iterations as the user decides to sum up the different template spectra until a good fitting of the spectral lines has been made to the observed spectrum. 

The output of the program after the execution contains the created spectrum (wavelength and intensity) and the approximate percentage of each of the stellar population inside the stellar system. Since the stellar populations are well documented the output will also contain the metallicity of each of them so that further analysis can be made upon STARLIGHT's results.

First, one must prepare the observed spectrum before running STARLIGHT, the spectrum has to be wavelength and flux calibrated, taking into account the bad-pixel removal. Very importantly in the context of mass analysis, the spectrum has to be extinction corrected so that the units of flux relate properly to the units if the templates in STARLIGHT.     

The extinction correction for our observed spectrum is given by

\begin{equation}
f_{obs}(\lambda)=f_{int}(\lambda)10^{-0.4A_{\lambda}}
\end{equation}

Where $A_{\lambda}=0.213$ in the I filter around $8000 \textrm{\AA}$, around the wavelength range of our spectrum. 

On our case, we have to multiply by a factor of 1.216746 the intensity of the spectrum for the flux calibration to be made. After we apply the extinction correction to the spectrum and create an ASCII table with the wavelength, intensity and error columns, it is now ready to be processed with STARLIGHT as we can see in the following figure:

\begin{figure}[H]
\centering
\includegraphics[width=10cm]{images/extinction.png}
\caption[Extinction Correction]{This figure shows an integrated spectrum of the central region of Omega Centauri before and after the extinction correction is applied. The black line has the original flux values and the black line has the corrected flux, that is, the flux that would be observed if there wasn't any interstellar medium that obscures the light coming from the object.}
\end{figure}

Before running STARLIGHT one must assure that the wavelength range is correctly specified in the configuration file that also includes the database of the template spectra and the bad data organized in a mask file. When all of these is ready it is straightforward to run STARLIGHT with the following command:

\begin{center}
./StarlightChains\_v04.exe $<$ Omega\_cen.in
\end{center}

The synthetic spectrum and the original one look like this:

\begin{figure}[H]
\centering
\includegraphics[width=10cm]{images/comparison.png}
\caption[Synthetic spectrum of STARLIGHT]{Synthetic spectrum of Starlight in red, shifted in the y axis for doing the comparison with the original spectrum of Omega Centauri in blue.}
\end{figure}
 
Now, besides the synthetic spectrum, the output file contains some useful results that one can use to calculate the mass of the stellar system. In our case, the relevant parameter that STARLIGHT gives is the stellar mass parameter given by:

\begin{equation}
Mcor\_tot = 3.29446 \times 10^{7}
\end{equation}

And using the formula:

\begin{equation}
M_{s}=Mcor\_tot\times10^{-17}\times4\pi d^{2}\times\left(3.826\times10^{33}\right)^{-1}
\end{equation}

Where $d$ is the luminosity distance in cm, yields a stellar mass of $M_{\star}=243.462M_{\odot}$

This mass is the stellar mass contained in the detection area (that in our set up configuration in OPD ends up to be $A_{D}=0.36\,pc^{2}$) of the integrated spectrum that we analysed with STARLIGHT so if we want to calculate the whole stellar mass of the Globular system we must extrapolate this result to its whole effective area, noting that this will increase the error of the calculation.

If we take the cluster's tidal radius of 40' and it's distance to the sun of $4808.39\,pc$ using a distance modulus of 13.41, then the total effective area (where the stellar mass could be calculated using stellar population synthesis) is $A_{OC}=9833.8\,pc^{2}$. 

Finally, the total stellar mass of the Cluster using this technique can be calculation using:

\begin{equation}
M_{s T} = N \times M_{\star}
\end{equation}

Where N is the number of detection areas within the total effective area of Omega Centauri ($A_{OC}/A_{D}$) of about 31844.8. So that our calculation of the stellar mass is finally:

\begin{equation}
M_{s T} = 6.61 \times 10^{6}M_{\odot}
\end{equation}
 
This result is actually higher than some values  of the dynamical mass found in the literature:

\begin{table}[H]
\begin{center}
\begin{tabular}{| c | c| }
    \hline
    \textbf{Article} & \textbf{Mass} ($M_{\odot}$) \\ \hline
    Mandushev et al. 1991 & $2.4 \times 10^{6}$  \\ \hline
    Pryor \& Meylan & $3.98 \times 10^{6}$  \\ \hline
    Meylan et al. 1995 & $5.1 \times 10^{6}$  \\ \hline
    Majewski et al. 2000 & $5.1 \times 10^{6}$  \\ \hline
    Van de Ven et al. 2006 & $2.5 \times 10^{6}$  \\ \hline
    Cassini et al. 2009 & $3.0 \times 10^{6}$  \\ \hline
    Valcarce \& Catelan, 2011 & $3.0 \times 10^{6}$  \\ \hline
    Jalali et al. 2011 & $2.5 \times 10^{6}$  \\
    \hline
  \end{tabular} 
\caption[Mass Omega Centauri]{Reported values of Omega Centauri's dynamical mass}
\end{center}
\end{table}

The stellar mass should in principle, by smaller or at least equals to the dynamical mass, this discrepancy in our first approach to the mass determination is probably due to errors given by the extrapolation of the results of the detection area to the whole area of the cluster, because our detection area was very small ($\sim 0.2 \, arcmin^{2}$) compared to the cluster's size of more than $6,000 \, arcmin^{2}$. Still, the stellar population technique is consistent with the order of magnitude of the cluster's mass previously reported.  


\begin{figure}[H]
\centering
\includegraphics[width=15cm]{images/Starlight_1.png}
\caption[En]{Tt.}
\end{figure}

\begin{table}[H]
\begin{center}
\begin{tabular}{| c| c| c| c| c| c| c|}
    \hline
    \textbf{Experiment} & $\mathbf{\beta}$ & $\mathbf{a_{dm}} (pc)$ & $\mathbf{a_{s}} (pc)$ & $\mathbf{M_{dm}}$ ($M_{\odot}$) & $\mathbf{M_{s}}$ ($M_{\odot}$) & $\mathbf{\Gamma}$\\ \hline
	Fix $a_s$ &	$0.95$ &	$58.0$ &	$2.23$ &	$8 \times 10^{4}$ &	$6.61 \times 10 ^{6}$ &	$2.3$\\ \hline
	Full &	$0.96$ &	$7.36$ &	$50.0$ &	$1.35 \times 10^{6}$ &	$6.61 \times 10^{6}$ &	$1.88$\\ \hline
  \end{tabular} 
\caption[It]{Ie}
\end{center}
  
\end{table}


\begin{figure}[H]
\centering
\includegraphics[width=15cm]{images/Starlight_2.png}
\caption[En]{Tct.}
\end{figure}

\begin{table}[H]
\begin{center}
\begin{tabular}{| c| c| c| c| c| c| c|}
    \hline
    \textbf{Experiment} & $\mathbf{\beta}$ & $\mathbf{a_{dm}} (pc)$ & $\mathbf{a_{s}} (pc)$ & $\mathbf{M_{dm}}$ ($M_{\odot}$) & $\mathbf{M_{s}}$ ($M_{\odot}$) & $\mathbf{\Gamma}$\\ \hline
	Fix $a_s$ &	$0.88$ &	$56.98$ &	$2.23$ &	$8 \times 10^{4}$ &	$6.61 \times 10 ^{6}$ &	$1.6$\\ \hline
	Full &	$0.96$ &	$7.6$ &	$12.0$ &	$6.88 \times 10^{5}$ &	$6.61 \times 10^{6}$ &	$0.9$\\ \hline
  \end{tabular} 
\caption[It]{Ie}
\end{center}
  
\end{table}

By taking an error bar of 25\% we do the same procedures but with the smallest value of the mass that the error bar allows us to use, in this case the value of the stellar mass is $M_{s}=4.9 \times 10^{6} M_{\odot}$

\begin{figure}[H]
\centering
\includegraphics[width=15cm]{images/Starlight_25_10.png}
\caption[En]{Tct.}
\end{figure}

\begin{table}[H]
\begin{center}
\begin{tabular}{| c| c| c| c| c| c| c|}
    \hline
    \textbf{Experiment} & $\mathbf{\beta}$ & $\mathbf{a_{dm}} (pc)$ & $\mathbf{a_{s}} (pc)$ & $\mathbf{M_{dm}}$ ($M_{\odot}$) & $\mathbf{M_{s}}$ ($M_{\odot}$) & $\mathbf{\Gamma}$\\ \hline
	Fix $a_s$ &	$0.88$ &	$56.98$ &	$2.23$ &	$9 \times 10^{4}$ &	$4.9 \times 10 ^{6}$ &	$1.1$\\ \hline
	Full &	$0.92$ &	$7.26$ &	$37.98$ &	$1.35 \times 10^{6}$ &	$4.9 \times 10^{6}$ &	$0.72$\\ \hline
  \end{tabular} 
\caption[It]{Ie}
\end{center}
  
\end{table}


\begin{figure}[H]
\centering
\includegraphics[width=15cm]{images/Starlight_25_12.png}
\caption[En]{Tct.}
\end{figure}

\begin{table}[H]
\begin{center}
\begin{tabular}{| c| c| c| c| c| c| c|}
    \hline
    \textbf{Experiment} & $\mathbf{\beta}$ & $\mathbf{a_{dm}} (pc)$ & $\mathbf{a_{s}} (pc)$ & $\mathbf{M_{dm}}$ ($M_{\odot}$) & $\mathbf{M_{s}}$ ($M_{\odot}$) & $\mathbf{\Gamma}$\\ \hline
	Fix $a_s$ &	$0.88$ &	$56.98$ &	$2.23$ &	$9 \times 10^{4}$ &	$4.9 \times 10 ^{6}$ &	$0.82$\\ \hline
	Full &	$0.98$ &	$13.26$ &	$31.98$ &	$1.98 \times 10^{6}$ &	$4.9 \times 10^{6}$ &	$0.66$\\ \hline
  \end{tabular} 
\caption[It]{Ie}
\end{center}
  
\end{table}

\chapter{Conclusions}

The main question regarding the mass modelling of $\omega$ Centauri is whether the cluster has any significant dark matter content or not so we focus our discussion on the context of this relevant question. The first point to be addressed here is the high dispersion of the stellar and dark matter masses in our results that can go from $8 \times 10^{4} M_{\odot}$ in the lowest case to $3.4 \times 10^{6} M_{\odot}$ in the highest case, in general this is a very high mass spectrum so the details of each of the models must be checked carefully.

Although the dispersion in our results is high, we note certain trends that suggest that the cluster has dark matter and it is comparable to the stellar mass of the system. One of this trends of our results is the fact that all of our models yield a dark matter mass different from zero ($M_{dm}\neq 0$) and ranging over a big spectrum of masses just like the stellar mass does. Although in most cases (all but one) the mass associated with dark matter is higher than the stellar mass ($M_{dm}\leq M_{s}$), their values are comparable to each other which is an important clue to understanding why the dark matter content of the cluster has not been observed before (if dark matter is in fact present in the cluster).

Regarding the scanlengths, we note that when we use a fixed value of the stellar scalength (found with photometry procedures), the scalength associated with the dark matter component is higher than the scalength associated with the stellar mass ($a_{dm} > a_{s}$), on the other hand, if we let the stellar scalength as a free parameter in the models, the results are the opposite ($a_{s} > a_{dm}$). We suggest that the models in which we set $a_{s}$ as a fixed value are the preferred models because we used the value reported by high quality photometric results and we used a direct and trustworthy method to calculate this value. So we come to the conclusion that in general, our best results yield a higher dark matter scalength than a stellar scalength which suggests that the dark matter halo of the cluster is more diluted than the halo associated to the baryonic matter.

In the case where we assume that the globular cluster's mass is only given by the stars (that is, no dark matter whatsoever), then our models reproduce well the reported values of dynamical mass and stellar scalength reported in previous results. This could suggest that $\omega$ Centauri does not have dark matter but this could be a blinded suggestion since a comparable or smaller value of dark matter could go unnoticed in the total mass of the system and could not be taken into account by the photometric and even radial velocity measurements.   

The anisotropy parameter doesn't really give us any important information because its values are too disperse to show any special trend, even though in most cases, the value is very small. As mentioned in chapter 4, we did not take the case where $\beta = 0$ because we wanted a more general form of our modelling. Our results suggest then that the determination of this value must be made using other methods regarding the proper motions of the stars in the cluster to have a better idea of its real value because we can't conclude anything about it with our results. The mass-to-light ratio on the other hand, reached values that are very close to the reported values in the literature, and ranged from 1.0 to 2.5 as a coincidence to the value found by Van de Ven et al. 2005.

But besides the ranging values of the fitted parameters, we found a very interesting trend in our results. In general, if we exclude the innermost region of the cluster by taking out the two first radial bins, we find that the models give us a higher dark matter mass ($M_{dm}^{(10 bins)}>M_{dm}^{(12 bins)}$) by a factor of 3.15 so that for a proper modelling of the cluster as a whole, the effects of a massive central object such as an intermediate mass black hole should be taken into account. The same effect is not observed in the stellar mass, where the number of radial bins does not really affect the trend.

Now, the results of our last procedures regarding a fixed stellar mass (found using stellar population synthesis with Starlight) gave us a very high value for the stellar mass that was actually bigger than most reported values so we did not give these results the same degree of confidence as we did to the other results. As it was expected, the dark matter mass found using these procedures was much smaller than the stellar mass for the fitting to yield a small $\chi^{2}$. Even with a small value for $M_{dm}$, the fitting was not very accurate so we didn't focus our conlusions on these results.

Finally, our most important conclusion is that using all of our results and taking into account the previous work that has be done upon $\omega$ Centauri, we suggest that if the cluster has any dark matter, then it would be close to its stellar mass or even smaller and it should be in the range from $5 \times 10^{5} M_{\odot}$ to $1.6 \times 10^{6} M_{\odot}$ that in any case would mean a significant fraction of the total dynamical mass of the cluster. 

This result deepens the questions regarding the origin of $\omega$ Centauri. A significant dark matter component could be a clue to verifying the theory that this object is actually the remnants of a nucleus of a dwarf galaxy absorbed by the Milky Way because as mentioned by Toshio et. al 2003, dwarf elliptical galaxies have big dark matter halos that even in a violent event (such as the accretion to a massive galaxy) would not be completely swept away form their host galaxy. If this theory is not correct and $\omega$ Centauri is in fact a globular cluster, then the presence of dark matter would mean that the theories of formation of these types of stellar systems should take into account the possibility that they were formed inside their own dark matter halos and thus the galaxy formation and evolution should be studied in concordance with these theories. 

\section{Future improvements}


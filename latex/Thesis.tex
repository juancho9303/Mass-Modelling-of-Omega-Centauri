%% ----------------------------------------------------------------
%% Thesis.tex -- MAIN FILE (the one that you compile with LaTeX)
%% ---------------------------------------------------------------- 

% Set up the document
\documentclass[a4paper, 11pt, oneside]{Thesis}  % Use the "Thesis" style, based on the ECS Thesis style by Steve Gunn
\graphicspath{Figures/}  % Location of the graphics files (set up for graphics to be in PDF format)

%\usepackage[T1]{fontenc}

% Include any extra LaTeX packages required
\usepackage[square, numbers, comma, sort&compress]{natbib}  % Use the "Natbib" style for the references in the Bibliography
\usepackage{verbatim}  % Needed for the "comment" environment to make LaTeX comments
\usepackage{vector}  % Allows "\bvec{}" and "\buvec{}" for "blackboard" style bold vectors in maths
\usepackage{float}
\usepackage{multirow}
\usepackage{amsmath, amsthm, amssymb, pifont}
\usepackage{caption}
%\usepackage[labelfont=bf]{caption}
\captionsetup{
format = plain,
font = footnotesize,
figurename = FIGURE,
tablename = TABLE,
labelfont = {sc,bf}
}
%\usepackage[labelfont=bf,labelsep=space]{caption}
\hypersetup{urlcolor=blue, colorlinks=true}  % Colours hyperlinks in blue, but this can be distracting if there are many links.
\newcommand{\xmark}{\ding{55}}%
\newcommand{\chulo}{\ding{52}}%

%%gdfgdgd ----------------------------------------------------------------
\begin{document}
\frontmatter      % Begin Roman style (i, ii, iii, iv...) page numbering

% Set up the Title Page
\title  {Mass Modelling of Globular Cluster $\mathbf{\omega}$ Centauri}
%\authors  {\texorpdfstring
%            {\href{juancho9303@gmail.com}{Juan Manuel Espejo Salcedo}}
%            {Juan Manuel Espejo Salcedo}
%            }
            
\authors  
            {{Juan Manuel Espejo Salcedo}}
            
           
\addresses  {\groupname\\\deptname\\\univname}  % Do not change this here, instead these must be set in the "Thesis.cls" file, please look through it instead
\date       {\today}
\subject    {}
\keywords   {}
%\begin{center}
%\includegraphics[scale=0.5]{Escudo-UdeA.png}
%\end{center} 

\maketitle
% ----------------------------------------------------------------

\setstretch{1.3}  % It is better to have smaller font and larger line spacing than the other way round

% Define the page headers using the FancyHdr package and set up for one-sided printing
\fancyhead{}  % Clears all page headers and footers
\rhead{\thepage}  % Sets the right side header to show the page number
\lhead{}  % Clears the left side page header

\pagestyle{fancy}  % Finally, use the "fancy" page style to implement the FancyHdr headers

%% ----------------------------------------------------------------
% Declaration Page required for the Thesis, your institution may give you a different text to place here
%\Declaration{

%\addtocontents{toc}{\vspace{1em}}  % Add a gap in the Contents, for aesthetics

%I, AUTHOR NAME, declare that this thesis titled, `THESIS TITLE' and the work presented in it are my own. I confirm that:

%\begin{itemize} 
%\item[\tiny{$\blacksquare$}] This work was done wholly or mainly while in candidature for a research degree at this University.
 
%\item[\tiny{$\blacksquare$}] Where any part of this thesis has previously been submitted for a degree or any other qualification at this University or any other institution, this has been clearly stated.
 
%\item[\tiny{$\blacksquare$}] Where I have consulted the published work of others, this is always clearly attributed.
 
%\item[\tiny{$\blacksquare$}] Where I have quoted from the work of others, the source is always given. With the exception of such quotations, this thesis is entirely my own work.
 
%\item[\tiny{$\blacksquare$}] I have acknowledged all main sources of help.
 
%\item[\tiny{$\blacksquare$}] Where the thesis is based on work done by myself jointly with others, I have made clear exactly what was done by others and what I have contributed myself.
%\\
%\end{itemize}
 
%Signed:\\
%\rule[1em]{25em}{0.5pt}  % This prints a line for the signature
 
%Date:\\
%\rule[1em]{25em}{0.5pt}  % This prints a line to write the date
%}
%\clearpage  % Declaration ended, now start a new page

%% ----------------------------------------------------------------
% The "Funny Quote Page"
\pagestyle{empty}  % No headers or footers for the following pages

\null\vfill
% Now comes the "Funny Quote", written in italics
\textit{``We are just and advanced breed of monkeys on a minor planet of a very average star. But we can understand the Universe. That makes us something very special.''}

\begin{flushright}
Stephen Hawking
\end{flushright}

\vfill\vfill\vfill\vfill\vfill\vfill\null
\clearpage  % Funny Quote page ended, start a new page
%% ----------------------------------------------------------------

% The Abstract Page
\addtotoc{Abstract}  % Add the "Abstract" page entry to the Contents
\abstract{
\addtocontents{toc}{\vspace{1em}}  % Add a gap in the Contents, for aesthetics

The mass modelling and formation history of galaxies have always been very interesting topics of Astrophysics and Cosmology. When you follow this routes it is perhaps inevitable the need of studying stellar systems inside galaxies (such as Globular Clusters) because they are present all along the way of the history, formation and structure of galaxies themselves. This thesis is intended to show our work on mass models of the local Globular Cluster $\omega$ Centauri with our data obtained in OPD observatory in Brazil and a recompilation of a good part of all available data of the cluster. By using stellar spectra of cluster members we compute radial velocities of the stars to obtain information about the velocity dispersion profile thus obtaining information about the potential well responsible for the dynamics of those individual stars. With this information, aside the mass estimations given by the photometry results, and in the context of the Jeans equations we can build mass models of the cluster looking for insight on the amount of dark matter present in this kind of structures, if dark matter is present at all.

}

\clearpage  % Abstract ended, start a new page
%% ----------------------------------------------------------------

\setstretch{1.3}  % Reset the line-spacing to 1.3 for body text (if it has changed)

% The Acknowledgements page, for thanking everyone
\acknowledgements{
\addtocontents{toc}{\vspace{1em}}  % Add a gap in the Contents, for aesthetics

I would like to thank my advisor and friend Juan Carlos Mu\~noz Cuartas for his constant efforts to guide me through this journey, not only in the academic part but also by giving me life lessons and tips to become a better professional and astronomer, his support, guidance and good will to help me with all my questions were a fundamental part of this work. His efforts came long before we started this project because he has always done everything he can to bring excellency to our academic program. I would also like to thank my professor and friend Jorge Zuluaga Callejas, whose work gave me the inspiration to study this beautiful career and because he has been part of my academic achievements from the very first moment I started my undergraduate studies, he has always shown great support for me and his efforts have set the high bar to professionals in Colombia that are truly committed to their jobs and bring knowledge to society.

Of course, I would like to thank my parents Hernando Espejo and Doris Salcedo, because everything that I have ever done including this thesis is because they have given me all their love and support and because they are my inspiration to make them feel proud. To my brother Leonel Ricardo Espejo, because he has been a fundamental role model in many aspects of my life. To many of my colleagues, friends and classmates such as Jorge Andr\'{e}s Villa, Nicol\'{a}s Gomez, Jaime Alvarado, Cesar Arroyo, Luis Fernando Quiroga, Daniel Arbelaez, Alejandro Guerra, Diego Bar\'{o}n, Fabio Cardona, Valentina Montoya, Denis Alvarez, Laura Arboleda, Yordan Arias, Carlos Boada and many more (if I typed all the names there would be too many) who gave me advise and helped me with theoretical and computational questions that were always present throughout this work and also because the fun and memorable moments that I had with them were an important part of the months I was working on this thesis.

To my basketball teammates in and out the court because they were another important aspect of my life as they kept me healthy and stable to be always willing to study and manage my time the best possible way. To all the cited authors who kindly helped science with their knowledge and efforts and because they were always willing to help me with any question regarding their data or their articles. To the LNA (Laborat\'orio Nacional de Astrof\'isica) for their financial support and their kind lodging that allowed us to acquire some of the data used in this thesis with the project I-034 in the OPD observatory.

To the University of Antioquia, the city of Medell\'{i}n, the professors and staff that were part of my formation as an astronomer and gave me the proper education to get me ready for my first serious research project, a project that I really enjoyed working on\ldots

}
\clearpage  % End of the Acknowledgements
%% ----------------------------------------------------------------

\pagestyle{fancy}  %The page style headers have been "empty" all this time, now use the "fancy" headers as defined before to bring them back

%  \renewcommand{\contentsname}%
%    {Whatever}%

%% ----------------------------------------------------------------
\lhead{\emph{Contents}}  % Set the left side page header to "Contents"
\tableofcontents  % Write out the Table of Contents

%% ----------------------------------------------------------------
%\lhead{\emph{List of Figures}}  % Set the left side page header to "List if Figures"
\listoffigures  % Write out the List of Figures

%% ----------------------------------------------------------------
\lhead{\emph{List of Tables}}  % Set the left side page header to "List of Tables"
\listoftables  % Write out the List of Tables

%% ----------------------------------------------------------------
%\setstretch{1.5}  % Set the line spacing to 1.5, this makes the following tables easier to read
%\clearpage  % Start a new page
%\lhead{\emph{Abbreviations}}  % Set the left side page header to "Abbreviations"
%\listofsymbols{ll}  % Include a list of Abbreviations (a table of two columns)
%{
% \textbf{Acronym} & \textbf{W}hat (it) \textbf{S}tands \textbf{F}or \\
%\textbf{LAH} & \textbf{L}ist \textbf{A}bbreviations \textbf{H}ere \\
%
%}

%% ----------------------------------------------------------------
%\clearpage  % Start a new page
%\lhead{\emph{Physical Constants}}  % Set the left side page header to "Physical Constants"
%\listofconstants{lrcl}  % Include a list of Physical Constants (a four column table)
%{
% Constant Name & Symbol & = & Constant Value (with units) \\
%Speed of Light & $c$ & $=$ & $2.997\ 924\ 58\times10^{8}\ \mbox{ms}^{-\mbox{s}}$ (exact)\\
%
%}

%% ----------------------------------------------------------------
%\clearpage  %Start a new page
%\lhead{\emph{Symbols}}  % Set the left side page header to "Symbols"
%\listofnomenclature{lll}  % Include a list of Symbols (a three column table)
%{
% symbol & name & unit \\
%$a$ & distance & m \\
%$P$ & power & W (Js$^{-1}$) \\
%& & \\ % Gap to separate the Roman symbols from the Greek
%$\omega$ & angular frequency & rads$^{-1}$ \\
%}
%% ----------------------------------------------------------------
% End of the pre-able, contents and lists of things
% Begin the Dedication page

\setstretch{1.3}  % Return the line spacing back to 1.3

\pagestyle{empty}  % Page style needs to be empty for this page
\dedicatory{Dedicated to my parents, whose love and support are my biggest motivation\ldots}

\addtocontents{toc}{\vspace{2em}}  % Add a gap in the Contents, for aesthetics


%% ----------------------------------------------------------------
\mainmatter	  % Begin normal, numeric (1,2,3...) page numbering
\pagestyle{fancy}  % Return the page headers back to the "fancy" style

% Include the chapters of the thesis, as separate files
% Just uncomment the lines as you write the chapters

\lhead{\emph{Introduction}}
\chapter{Introduction}

As the oldest known subunits of our Galaxy, Globular Clusters are our primary "fossils" from its early evolution, and they may hold the key to understanding its formation. For example, the amount of dark matter that they could or not have may be an important clue to the understanding of the formation of the Milky Way. Despite the large amount of observational data and the advances in theoretical work, still, there is not yet a convincing model to describe the formation of these ancient structures. 

Two broad types of possibilities have been considered: one is that the globular clusters were the very first condensed systems to form in the early universe, and the second is that they originated in larger star-forming systems that later merged to form the present galaxies. 

The first possibility includes the suggestion of Peebles \& Dicke (1968) that the globular clusters were formed by Jeans fragmentation in the early universe, this possibility is in concordance with the current scenario of galaxy formation in which galaxies are formed in to the deepest regions of the gravitational potential well provided by dark matter halos. This explanation has been received with great interest by the community since it not only fits within the hierarchical scenario of structure and galaxy formation, but also may help to understand some of the open problems in the standard model, such as the abundance of low mass structures (Klypin et. al. 1999). However, evidence has been found against this scenario. For example, Odenkirchen et al. (2003) have found tidal tails surrounding globular clusters, something that is not expected if globular clusters form and reside inside extended dark matter halos.

Regarding the second possibility, Fall \& Ries (1985, 1988) proposed the formation of globular clusters as a response to thermal instabilities in the hot gaseous halos of massive galaxies. The Fall \& Rees hypothesis has been popular among theorists who have used it to predict the characteristic properties of globular clusters, although it has proven difficult to justify the assumed thermal behaviour of the cluster-forming gas clouds. This scenario presents a problem since there are observations that suggest that very low mass galaxies, not massive enough to host a hot gaseous halo, may also have their own globular clusters.

Not only the formation of these structures is puzzling, their composition is also a challenging problem because some authors say that these structures do not contain any dark matter contributions and that their gravitational stability can be explained completely with the baryonic matter inside them. Conroy et. al. (2011) have used density profiles to argue against the presence of dark matter inside globular clusters, while Ibata et. al. (2013) have found that under general conditions it could be possible to find significant fractions of dark matter in globular clusters. Modelling the mass content of globular clusters in the galaxy would help to disentangle the mechanism driving its formation process. Mass modelling would allow to study the mass distribution in the inner region of globular clusters, determining the dominant components (Breddels M. A. et.al. 2013, Adams J. J. et. al., 2012, van den Bosch R.C.E. et. al. 2006) providing light on the problem of the origin of globular clusters.

Our aim in this project is to build our own mass model for the Globular Clusters using some data observed in OPD observatory in May 2014 so that we can be able to discuss and find new evidence on the existence or absence of dark matter in them according to these results. 

With our set of data, we do the preliminary reduction and analysis of photometric and spectroscopic data, including the wavelength and flux calibration of the spectra. The photometric data will show us the mass to light ratio of the Globular Clusters thus giving us information about the baryonic mass content of the clusters. On the other hand, the spectroscopic data will give us information about the radial velocity of the stars that will provide us the statistical dispersion of velocities in the inner region of the clusters thus giving us information about the potential well in the clusters. This procedure is done using the Radial Velocity Package of IRAF called \textit{RVSAO} which uses cross correlation techniques in the Fourier transform of templates and scientific spectra to infer the oppler Broadening of the emission and/or absorption lines in our data and calculating the radial velocities. 

After this analysis has been made upon all the observational data we can start the theoretical analysis including simulations of N-body and study of the initial conditions that will preserve the spherical symmetry of the clusters, taking into account our results on the dominant components of the clusters. 

The bulk properties of GCs, with the possible exception of their innermost regions, can be modelled using the collisionless Boltzmann equation (Binney \& Tremaine 1987), from which the statistical properties of the velocity distribution of stars can be derived. In particular, one can derive a formulation for the velocity dispersion tensor that, in the isotropic, non rotating case, reduces to a scalar quantity. This quantity can be determined using the appropriate Jeans equation. This calculation requires knowledge of the distribution function (DF) that determines the number of stars in a given region of space, and the gravitational potential.

The best fit for our data with the theoretical assumptions will tell us how mass is distributed in the clusters and it will allow us to conclude if there is a significant contribution of dark matter to the potential well that provides the stars the observed velocities. As mentioned before, the Mass Modelling will let us have a good insight into the problem of the formation of these structures.



 % Introduction

\lhead{\emph{Theoretical Framework}} 
\chapter{Theoretical Framework}

- Cumulos globulares (definiciones, propiedades, formacion y evolucion, poblaciones estelares, etc)

\section{Globular Clusters}

Typical galaxies all around the Universe hold different structures such as stellar systems of between $ 10^{2} $ and $ 10^{6} $ stars which orbit their galactic core . We call these interesting systems star clusters and they are basically divided into two main types: Open Clusters and \textbf{Globular Clusters}.

Globular clusters are very massive stellar systems that can contain from thousands to millions of stars in a nearly spherical distribution spread over a volume of several tens to about 200 light years in diameter. These stellar systems are composed of old stars and they do not contain gas or dust. 

The average star density in a Globular Cluster is about 0.4 stars per cubic parsec. In the dense center of the cluster, the star density can increase from 100 to 1000 per cubic parsec. However, even in the center of clusters, there is still plently of space between the stars

A way of analysing globular clusters is to use Colour-Magnitude diagrams. A colour-magnitude diagram is a plot of the apparent magnitudes of the stars in a cluster against their colour indices. Globular clusters nearly all have very similar colour-magnitude diagrams

Globular clusters revolve about the nucleus of a galaxy on orbits of high eccentricity and high inclination to the galactic plane. About a third of globular clusters are concentrated around the galactic center. A typical cluster has a period of revolution around the order of $ 10^{8} $ years. A cluster spends most of its time far from the center of a galaxy, and so most of them can, and have been discovered in the spaces between galaxies.



\subsection{Photometric Properties}

\section{Stellar System Dynamics}

(we focus on globular clusters)

\subsection{Colisionless Systems}

\begin{equation}
\dfrac{df}{dt}=0
\end{equation}


\section{Scenario and Observations}

\section{Simulations}
 % Theoretical Framework 

\lhead{\emph{Obervational Procedures}}
%\usepackage{subfig}

\chapter{Observations and Analysis}

In order to study this problem about the dynamics of Globular Clusters in our galaxy we need scientific data that allows us to build a model that fits our observations. Under supervision of proffesor Juan Carlos Mu\~noz Cuartas and with three other undergraduate students from the University of Antioquia a trip to the OPD (Pico dos Dias Observatory) was made to Brazil in May 2014, besides the observational experience of the students, the main purpose of the trip was to get important data for this project. We needed two sets of data corresponding to spectra and photometric images of the Globular Clusters

The spectroscopic data allows us to determine the velocity dispersion profile in the inner region of globular clusters while the photometric data allows us to study the surface brightness distribution for them. We can use all of this information to infer the properties of the globular clusters' mass distribution in order to build complete dynamical models and therefore infer the amount of dark matter present in the globular clusters (if there is any).

\section{Observational Procedures}

Our stay in OPD consisted of two days in the main dome for the spectroscopic data (using the Perkin-Elmer (P\&E) telescope with a 1.6m mirror and the Cassegrain Spectrograph) and four days in a smaller dome for the photometric data in the IAG telescope with a 0.6m mirror. In the following photograph, the domes of the observatory that we used for our observations:

\begin{figure}[h]
\centering
\includegraphics[width=10cm]{images/opd.jpg}
\caption{OPD observatory seen from the air, the big dome was used for the spectroscopic data and the small dome at the low right part of the photo for the photometric data.}
\end{figure}

\subsection{Spectroscopic Data}

The first two days (May 14th and 15th) we took the spectroscopic data in the telescope P\&E with a diameter of 1.6m. The main instrument was the Cassegrain spectrograph with a CCD Ikon-L camera and Filters BVR. The software we used was the recently installed software TCSPD which is built in a LabView environment for Windows (2010). Here's a photo of the telescope from inside the dome:

\begin{figure}[h]
\centering
\includegraphics[width=10cm]{images/opd-spectrograph.jpg}
\caption{Perkin-Elmer telescope in the main dome in OPD used for the spectroscopic observations}
\end{figure}

We made the observations of dome flats, bias frames, comparison lamp frames, calibration stars and certain globular clusters of the milky Way organized by the best observation times using Simbad and Stellarium for the estimations of the coordinates and times respectively. We needed to keep an order of the observations to make the most of our observation time in OPD so we decided to organize our Globular Clusters in different groups or "chunks":

\begin{figure}[h]
\centering
\includegraphics[width=10cm]{images/9.png}
\caption{Organized globular Clusters in groups for the proper times}
\end{figure}

Now, our set up configuration for the spectrograph was the following:

On May 14th, a diffraction grating of 900 lines per mm, a CCD IkonL and the central wavelength for the observations of 8500 Angstroms (with possibility of rotation of the slit 90°, +45° and -45°).we used the slit of 2.52" and obtained data for the globular clusters: NGC-5020, NGC-5272, NGC-4833, NGC-4590, NGC-5139, NGC-5286, NGC-6752, NGC-6397, NGC-6723, NGC-6715 and NGC-6541 using exposition times of 600 and 900 seconds. We also observed the calibration stars: HR-4963 and HR-4468 with 7 and 5 seconds. As it was the first day, we needed to be very careful in calibrating our instruments on order to have the objects in the right focus, we also made the rotation of the slit to use all the diffraction angles of the observations and our comparison lamps were of Ne-Ar.

On May 15th, we used the slit of 3.0", and used a central wavelength of 5500 Angstroms. This time we observed the following objects: NGC-2802, NGC-5024, NGC-4590, NGC-5139, NGC-5286, NGC-5272, NGC-6362, NGC-6397, NGC-6723, NGC-6502, NGC-6541, NGC-7078, NGC-7099, the stars HR-4468 and HR-7950 and we also observed Mars for pedagogical reasons. We used pretty much the same exposition times than the day before, this time though, our comparison lamps were or He-Ar. All the data we took was in FITS format (Flexible Image Transfer System).

\subsection{Photometric Data}

The photometric data were acquired in the next four days (from May 16th to May 19th) in the 0.6m IAG telescope in OPD. We used the Johnson system for the different filters which were easily shifted with the given software in the control computers. Here a picture of the telescope from inside the dome:

\begin{figure}[h]
\centering
\includegraphics[width=8cm]{images/opd-photometry.jpg}
\caption{IAG telescope used for the photometric data}
\end{figure}

On May 16th, we took all the calibration images, consisting of 20 bias frames with an exposition time of 0,00001 seconds; also 22, 11, 11, 20 and 10 flat frames for the B,I,R,U,V filters respectively, their exposition times differed, for  U filter we took various frames of 60 and 30 seconds, for the B filter we took frames of 30 seconds each, 15s for I, 60s for R and 3s for V. We took our "focus" images to calibrate the instrument, and also various skyflats for all the filters. We targeted the following globular clusters and calibration stars in different filters: NGC5272, HR4961, NGC4590, NGC5139 AND NGC6397. The exposition time for the clusters was of 600 seconds and 2 and 4 seconds for the calibration star. 

May 17th was a terrible night for observations because the sky was too cloudy and the only useful data we could get were dome flats for the filters I,R and V that we could use instead of the bad dome flats of the first day. The reduction using the flats of another day are decent but this is not the ideal situation since mechanical movements of the instrument might slightly change its configuration and therefore it probably ends up with a reduction that is not the ideal one for science purposes.  

On May 18th we were more organized since we were getting familiar with the observations and therefore the data we got had little trouble in the upcoming analysis, even though the sky was clody at the end of the night. The science objects we observed were NGC5139, HR6308, NGC6723, NGC6541, NGC7078, HR7964 that were observed in the different filters. We got 20 bias frames, 14 dome flats in the vaious filters, but no skyflats. 

On May 19th we observed the Globular Clusters NGC5139, NGC4590, NGC6723, NGC6715, NGC6541, NGC6970, NGC5286, NGC639, NGC6541 and NGC6715, the calibration stars HR6386 and HR6386, 20 bias frames and flats for each filter.

\section{First step for Anaysis}

Our first goal in starting the analysis of all the relevant data was to organize all the images in order to reduce the time required to make the reductions. For every day the calibrations images, trash, calibration stars and objects were separated and they were given their correct names as they were in the headers and compared with the information sheets we filled at the time we were doing the observations. With the use of an account on the galaxy.udea.edu.co cluster, for proper and quicker analysis and safety of the data, all the files were correctly organized.

The next step was the reduction of all the images with the calibration files for each day, I started the photometric data to acquire certain skills in the use of IRAF because the reduction of the spectroscopic data was to be a little more complex and needed a deeper understanding of IRAF packages. 

I started with the cluster NGC-5139 ($ \Omega $ Centauri) because we got lots of data for that cluster in OPD and also because $ \Omega $ Centauri is a well known globular cluster since it is the largest in our galaxy and we can get a lot of information from the web. 

After the photometry of that cluster, the most relevant part of the reduction was to be made. The reduction and analysis of the spectroscopic data (May 14th and 15th), the methods for these reductions are quite special and are the most relevant part of the analysis because that is our most valuable information. The reduction was to be made very carefully because a good spectroscopic analysis depends upon a good reduction of the data. Just as with the photometric data, the first procedures were made for the Cluster NGC5139 to understand and master the techniques of the reduction and extractions.

\section{Photometry}

The photometry was made by the two traditional methods, PSF photometry and Aperture Photometry; even though the magnitudes calculated using both methods are quite different, the calibration constant between the two methods gave a good relation between them and made me trust the photometry results.

But first, the reduction of the data had to be done. The first step is to characterize the calibration images in order to see if there are any errors associated with the instrument or the way that the observations were made. By doing this we found that most of the flat-field images had brightness gradients in the corners and this was a problem we needed to correct because the increased value on the counts in these corners would affect the normalization of the super-flat that we would use to reduce the science data. Another systematic error that we found in all of our calibration and science frames was the prescence of a strange water-looking figure at the top left corner of them, although it can be removed with the correct reduction, it obviously affected the CCD sensitivity by the time of the observations. Also, some filters showed a higher sensitivity to this systematic errors but at the end, the photometry could be made in the best data so that the dirty images don't affect our results.

In order to see how the data would be affected by the systematic errors we just mentioned, we produced a composite image using three images with the filters U,V and R and we did the same with the flats in those filters, the resultas are shown in the following figure:
  
\begin{figure}[h]
  \centering
  \begin{minipage}[b]{0.45\textwidth}
    \includegraphics[width=\textwidth]{images/ngc_5139_dirty.png}
    \caption{Composite image of NGC5139 without being previously reduced}
  \end{minipage}
  \hfill
  \begin{minipage}[b]{0.45\textwidth}
    \includegraphics[width=\textwidth]{images/ruido.png}
    \caption{Composite image of the flats showing the noise that needs to be extracted}
  \end{minipage}
\end{figure}

What we can infer from these images is that the flat fields and the bias frames contain the same noise that the science data thus giving us a good result in the reduction.

Once all the characterization is made we can reduce our important data using IRAF following the conventional steps consisting of: 

building a super-bias

Hago un Superbias a partir de los bias que tenga
Le resto ese Superbias a todos los flats
Hago un Superflat para cada filtro
A esos superflats los divido por la media de cada uno
Las imágenes originales las divido por esos flats

\begin{figure}[h]
\centering
\includegraphics[width=8cm]{images/flat_I.png}
\caption{Normalized Superflat for the I filter}
\end{figure}




Photometry using Phot

\section{Spectroscopy}


\subsection{Spectroscopic Reduction}

- First we make a Superbias combining all the bias frames and then we subtract it from all the lamp, targets and flat field frames.

- It was important to analize the flats to see which ones are saturated, we consider that values over 65,000 counts (using implot) show saturated data. The ones that we could trust for May 14th were ten images called flats\_0012 to flats\_0021.

- The pre-superflat is made using the median given the number of images.

- We need to make a trimming in all images because there are some regions in the images that show unexpected luminosity, this is probably due to border errors in the camera or the obturator time of relaxation. The zones we decided to cut were:

[0-100] and [575 to the end]

- A critical step is the creation of a response function, this is made by collapsing the pre-superflat to one column using blkavg. The useful image for the creation of the Superflat is done by combining this column with blkrep. This gives us an image that's uniformly distributed in the dispersion axis with the following IRAF commands.

blkavg MasterFlat.fits[1:475,*] AvgFlatCols 475 1

blkrep AvgFlatCols AvgFlatColsMaster 475 1

- The pre-superflat is now divided by the response function we created (AvgFlatColsMaster) and this gives us the Superflat that we will use to reduce our data.

- Finally, the task we use to remove the cosmic rays is lacos, and it gives very accurate results, as it shows the "mask" image with the removed cosmic rays.

\subsection{Extraction}

Once the reduction is ready, we can proceed with the extraction of the spectra of the calibration stars and also the spectra of the stars in the clusters, this procedure is made with the task apall.

Taking special care of correctly choosing the background, and with the following parameter configuration:

b\_number: 100

background: fit

weight: vairance

saturate: 65215

rdnoise: 6

gain: 1

Interactively, one must choose very precisely the background regions to extract the spectrum and do the fitting routines with different orders until the best results are reached.

The extraction of the spectrum for the calibration lamps is done with apsum, which is very similar to apall.

\subsection{Wavelength Calibration}

The wavelength calibration is made many tasks of IRAF like Identify, Refspec and Dispcor. First, with identify I use the interactive window in IRAF to select some prominent lines in the spectrum and assign them their correct wavelength using the theoretic spectrum of the lamp. In this case our calibration lamps were Ne-Ar (for May 14th) and He-Ar (for May 15th)  and OPD observatory provided us the theoretic distribution of emission lines of them.

Using "m" to select the larger lines and typing the wavelength, the task creates a file stored in a new folder "database" with the pixels with their corresponding values in units of Angstroms. After that, the targets were to be calibrated with these files so it is necessary to edit their header to assign them the reference frames. It is enough to change the REFSPEC1 image header on each lamp file in order to do the wavelength calibration. 

The task that actually does the calibration on wavelength is dispcor, it is only necessary to run the task over all the targets with their own calibrated arc to get the calibrated spectrum which is the useful and important file to make the analysis of the width of the lines and their redshift.

\subsection{Flux Calibration}

The aim is to calibrate the CCD chip response, spectrograph+telescope throughput and allow for atmospheric extinction. The result is a spectrum as observed from outside the atmosphere with an ideal uniformly sensitive detector+telescope+spectrograph. Basically, what the flux calibration does is, it takes from a tabular compilation the energy distribution of the standard star, it corrects this energy distribution for wavelength-dependent atmospheric extinction, it compares it to the energy distribution of the observed spectrum and derives from such a comparison the function that gives the response of our system for every wavelength.

The flux calibration takes place in three parts: Calibrating from the standard star, calculating the sensitivity function of the instrument, and finally, applying the calibration to the spectra. We will use the task observatory to determine observatory parameters, standard to flux calibrate each standard star, and sensfunc to finally determine the wavelength response and the solution will be applied to the spectra by the task calibrate.

In the first part, the calibration is made with one of the stars that are already included in IRAF, there are many stars so there's quite a good amount of options to choose. So the first task is the task standard. The observatory parameter is specified as LNA which is in IRAF's database. 

\textbf{The task standard}

The task standard determines calibration pass-bands and writes them to a file called std. The thrick here is to specify the location of the the input extinction and flux calibration files. To do that, I edit the parameters of standard with the following routes:

Extinction file:                              onedstds\$/ctioextindt.dat

Directory containing calibration data:   onedstds\$ctionewcal/

Starname in calibration list:                l9239

Where I chose the Star l9239 because it has the spectral range that we use in our calibration Stars. And running the task interactively would be enough for this step.

\textbf{The task sensfunc}

Standard task just recorded response of each standard star so the next step is to put the results together and find a proper wavelength dependence of instrumental sensitivity and atmosphere transparency using the task sensfunc. It creates an image with a default name sens.0001. IRAF needs to have some general idea of atmospheric extinction before to start, so I set again extinct onedstds\$ /ctioextinct.dat.

Now, running the task interactively and taking into account that the function used to fit the instrumental response will be usually of very high order. A good idea is to use spline3 fitting (:function spline3) with some 20 pieces, i.e. (:order 20).
Finally q exists the sensfunc task and writes the sens.0001 image.

\textbf{The task calibrate}

The solution to each star to be calibrated is done with the task calibrate. Editting the parameters of calibrate to set the appropriate extinction table: extinct onedstds\$ /ctioextinct.dat would be enough for this purpose. The task is run over all the wavelength calibrated spectra which had their airmass and other parameters appropriately set by the eso.set procedure. And finally it gives the flux-calibrated spectra ready for the relevant analysis concerning radial velocities.

After the flux calibration, I notice that the extremes of the spectra have irregularities but that can be cut because they don't have any relevant information.

For the star calibration I cut from 0 to 45 and from 1860 to the end, using imcopy:

\begin{center}
imcopy flux\_calib\_star\_fits[45:1860,*] cut\_flux\_calib\_star.fits
\end{center}

In order to normalize the spectrum, first I find the maximum value in the spectrum using minmax and then I divide the whole image by this value.

Now, to create the Ascii table from the spectrum I need to first convert my image to a 1D image using the task scopy and setting format=onedspec

Now, with the image ready in 1D, I use the task wspectext to create the Ascii table like this:

\begin{center}
wspectext ready\_flux\_star.0001.fits normal\_cut\_flux\_star\_calib.txt
\end{center}

\section{RVSAO and radial velocity determination} % Observational Procedures
 
\lhead{\emph{Modelling}} 
\chapter{Modelling}

\section{Mass determination of Omega Centauri}

We used various techniques for the determination of the dynamic and stellar mass of NGC5139 that we use to model the mass more accurately. 

\subsection{Stellar Population Synthesis with Starlight}

The stellar mass content of Globular Clusters and Galaxies can be studied through the determination of the stellar populations inside those systems since we have clear knowledge about their photometric properties. If we have information about the amount of stars of a given type inside a stellar system, we can infer how much of the system's mass is given by these populations of stars. 

The determination of the stellar populations can be done using STARLIGHT, which is a Fortran-based program that fits an observed integrated spectrum (Omega Centauri in our case) with a model spectrum which is the sum of $N_{*}$ spectral components from a pre-defined and pre-processed set of base spectra. The program does as many iterations as the user decides to sum up the different template spectra until a good fitting of the spectral lines has been made to the observed spectrum. 

The output of the program after the execution contains the created spectrum (wavelength and intensity) and the approximate percentage of each of the stellar population inside the stellar system. Since the stellar populations are well documented the output will also contain the metallicity of each of them so that further analysis can be made upon STARLIGHT's results.

First, one must prepare the observed spectrum before running STARLIGHT, the spectrum has to be wavelength and flux calibrated, taking into account the bad-pixel removal. Very importantly in the context of mass analysis, the spectrum has to be extinction corrected so that the units of flux relate properly to the units if the templates in STARLIGHT.     

The extinction correction for our observed spectrum is given by

\begin{equation}
f_{obs}(\lambda)=f_{int}(\lambda)10^{-0.4A_{\lambda}}
\end{equation}

Where $A_{\lambda}=0.213$ in the I filter around $8000 \textrm{\AA}$, around the wavelength range of our spectrum. 

On our case, we have to multiply by a factor of 1.216746 the intensity of the spectrum for the flux calibration to be made. After we apply the extinction correction to the spectrum and create an ASCII table with the wavelength, intensity and error columns, it is now ready to be processed with STARLIGHT as we can see in the following figure:

\begin{figure}[H]
\centering
\includegraphics[width=10cm]{images/extinction.png}
\caption[Extinction Correction]{This figure shows an integrated spectrum of the central region of Omega Centauri before and after the extinction correction is applied. The black line has the original flux values and the black line has the corrected flux, that is, the flux that would be observed if there wasn't any interstellar medium that obscures the light coming from the object.}
\end{figure}

Before running STARLIGHT one must assure that the wavelength range is correctly specified in the configuration file that also includes the database of the template spectra and the bad data organized in a mask file. When all of these is ready it is staightforward to run STARLIGHT with the following command:

\begin{center}
./StarlightChains\_v04.exe $<$ Omega\_cen.in
\end{center}

The synthetic spectrum and the original one look like this:

\begin{figure}[H]
\centering
\includegraphics[width=10cm]{images/comparison.png}
\caption[Synthetic spectrum of STARLIGHT]{Synthetic spectrum of Starlight in red, shifted in the y axis for doing the comparison with the original spectrum of Omega Centauri in blue.}
\end{figure}
 
Now, besides the synthetic spectrum, the output file contains some useful results that one can use to calculate the mass of the stellar system. In our case, the relevant parameter that STARLIGHT gives is:

\begin{equation}
Mcor\_tot = 3.29446 \times 10^{7}
\end{equation}

And using the formula:

\begin{equation}
M_{\star}=Mcor\_tot\times10^{-17}\times4\pi d^{2}\times\left(3.826\times10^{33}\right)^{-1}
\end{equation}

Yields a stellar mass of $M_{\star}=243.462M_{\odot}$

This mass is the stellar mass contained in the detection area (that in our set up configuration in OPD ends up to be $A_{D}=0.36\,pc^{2}$) of the integrated spectrum that we analysed with STARLIGHT so if we want to calculate the whole stellar mass of the Globular system we must extrapolate this result to its whole effective area, noting that this will increase the error of the calculation.

If we take the cluster's tidal radius to be 45' (Trager et al. 1995) and it's distance to the sun of $4808.39\,pc$ then the total effective area (where the stellar mass could be calculated using stellar population synthesis) is $A_{OC}=12445.9\,pc^{2}$. 

Finally, the total stellar mass of the Cluster using this technique can be calculation using:

\begin{equation}
M_{\star T} = N \times M_{\star}
\end{equation}

Where N is the number of detection areas within the total effective area of Omega Centauri ($A_{OC}/A_{D}$) of about 31844.8. So that our calculation of the stellar mass is finally:

\begin{equation}
M_{\star T} = 6.2734 \times 10^{6}M_{\odot}
\end{equation}
 
This result is actually higher than some values  of the dynamical mass found in the literature:

\begin{table}[H]
\begin{center}
  \begin{tabular*}{0.55\textwidth}{@{\extracolsep{\fill} } |  c | c | }
    \hline
    \textbf{Paper} & \textbf{Mass} \\ \hline
    Van de Ven et al. 2008 & $\sim 2.5 \times 10^{6} M_{\odot}$ \\
    Mandushev et al. 1991 & $\sim 2.4 \times 10^{6} M_{\odot}$ \\
    Meylan et al. 1995 & $\sim 5.1 \times 10^{6} M_{\odot}$ \\
    Jalali et al. 2011 & $\sim 2.5 \times 10^{6} M_{\odot}$ \\
    \hline
  \end{tabular*}
\end{center} 
\caption[Mass Omega Centauri]{Reported values of Omega Centauri's dynamical mass}
\end{table}

The stellar mass should in principle, by smaller or at least equals to the dynamical mass, this discrepancy in our first approach to the mass determination is probably due to errors given by the extrapolation of the results of the detection area to the whole area of the cluster, because our detection area was very small ($\sim 0.2 \, arcmin^{2}$) compared to the cluster's size of more than $6,000 \, arcmin^{2}$. Still, the stellar population technique is consistent with the order of magnitude of the cluster's mass previously reported.  

\subsection{Modified Hernquist Model}

\subsection{Color-Magnitude diagrams}

 % Modelling

\lhead{\emph{Conclusions}} 
\chapter{Conclusions}

The main question regarding the mass modelling of $\omega$ Centauri is whether the cluster has any significant dark matter content or not so we focus our discussion on the context of this relevant question. The first point to be addressed here is the high dispersion of the stellar and dark matter masses in our results that can go from $8 \times 10^{4} M_{\odot}$ in the lowest case to $3.4 \times 10^{6} M_{\odot}$ in the highest case, in general this is a very high mass spectrum so the details of each of the models must be checked carefully.

Although the dispersion in our results is high, we note certain trends that suggest that the cluster has dark matter and it is comparable to the stellar mass of the system. One of this trends of our results is the fact that all of our models yield a dark matter mass different from zero ($M_{dm}\neq 0$) and ranging over a big spectrum of masses just like the stellar mass does. Although in most cases (all but one) the mass associated with dark matter is higher than the stellar mass ($M_{dm}\leq M_{s}$), their values are comparable to each other which is an important clue to understanding why the dark matter content of the cluster has not been observed before (if dark matter is in fact present in the cluster).

Regarding the scanlengths, we note that when we use a fixed value of the stellar scalength (found with photometry procedures), the scalength associated with the dark matter component is higher than the scalength associated with the stellar mass ($a_{dm} > a_{s}$), on the other hand, if we let the stellar scalength as a free parameter in the models, the results are the opposite ($a_{s} > a_{dm}$). We suggest that the models in which we set $a_{s}$ as a fixed value are the preferred models because we used the value reported by high quality photometric results and we used a direct and trustworthy method to calculate this value. So we come to the conclusion that in general, our best results yield a higher dark matter scalength than a stellar scalength which suggests that the dark matter halo of the cluster is more diluted than the halo associated to the baryonic matter.

In the case where we assume that the globular cluster's mass is only given by the stars (that is, no dark matter whatsoever), then our models reproduce well the reported values of dynamical mass and stellar scalength reported in previous results. This could suggest that $\omega$ Centauri does not have dark matter but this could be a blinded suggestion since a comparable or smaller value of dark matter could go unnoticed in the total mass of the system and could not be taken into account by the photometric and even radial velocity measurements.   

The anisotropy parameter doesn't really give us any important information because its values are too disperse to show any special trend, even though in most cases, the value is very small. As mentioned in chapter 4, we did not take the case where $\beta = 0$ because we wanted a more general form of our modelling. Our results suggest then that the determination of this value must be made using other methods regarding the proper motions of the stars in the cluster to have a better idea of its real value because we can't conclude anything about it with our results. The mass-to-light ratio on the other hand, reached values that are very close to the reported values in the literature, and ranged from 1.0 to 2.5 as a coincidence to the value found by Van de Ven et al. 2005.

But besides the ranging values of the fitted parameters, we found a very interesting trend in our results. In general, if we exclude the innermost region of the cluster by taking out the two first radial bins, we find that the models give us a higher dark matter mass ($M_{dm}^{(10 bins)}>M_{dm}^{(12 bins)}$) by a factor of 3.15 so that for a proper modelling of the cluster as a whole, the effects of a massive central object such as an intermediate mass black hole should be taken into account. The same effect is not observed in the stellar mass, where the number of radial bins does not really affect the trend.

Now, the results of our last procedures regarding a fixed stellar mass (found using stellar population synthesis with Starlight) gave us a very high value for the stellar mass that was actually bigger than most reported values so we did not give these results the same degree of confidence as we did to the other results. As it was expected, the dark matter mass found using these procedures was much smaller than the stellar mass for the fitting to yield a small $\chi^{2}$. Even with a small value for $M_{dm}$, the fitting was not very accurate so we didn't focus our conclusions on these results.

Finally, our most important conclusion is that using all of our results and taking into account the previous work that has be done upon $\omega$ Centauri, we suggest that if the cluster has any dark matter, then it would be close to its stellar mass or even smaller and it should be in the range from $5 \times 10^{5} M_{\odot}$ to $1.6 \times 10^{6} M_{\odot}$ that in any case would mean a significant fraction of the total dynamical mass of the cluster. 

This result deepens the questions regarding the origin of $\omega$ Centauri. A significant dark matter component could be a clue to verifying the theory that this object is actually the remnants of a nucleus of a dwarf galaxy absorbed by the Milky Way because as mentioned by Toshio et al. 2003, dwarf elliptical galaxies have big dark matter halos that even in a violent event (such as the accretion to a massive galaxy) would not be completely swept away form their host galaxy. If this theory is not correct and $\omega$ Centauri is in fact a globular cluster, then the presence of dark matter would mean that the theories of formation of these types of stellar systems should take into account the possibility that they were formed inside their own dark matter halos and thus the galaxy formation and evolution should be studied in concordance with these theories. 

\section{Future improvements}

There are many things to be improved in the mass modelling of the cluster to reach better results, whether it is changing the potential-density pair model or increasing the quality of the data that we used for this model in particular. But our results show that there are some particular improvements of our model that need to be done for an accurate and comprehensive modelling of this stellar system.

$\rightarrow$ As it was shown in our results, the centre of the cluster presents a slightly noticeable deviation from the theoretical curves which suggests that there is a massive body such as a black hole that affects the dynamics of the stars increasing their radial velocities. A further and more detailed analysis of the velocity dispersion of the cluster should include the meticulous study of the centre and include such body in the model.

$\rightarrow$ In order to make a better observational curve with more radial bins, the measurement of the radial velocities of more stars is necessary, because even with a relatively high number of stars ~3700 stars we were only able to construct 12 radial bins that we could give a high level of confidence. The observation of more individual radial velocities is important not only for the construction of the observational curve but also because the error associated with every bin is inversely proportional to the number of velocities used to construct the velocity dispersion on it. 

$\rightarrow$ The stellar population synthesis procedures that we did with Starlight are a very interesting way to determine the stellar mass of the system, but our integrated spectrum didn't have enough quality to give us trustworthy results. For a next or future work, these procedures have to be made using a higher number of high quality spectra in order to have a significant statistical sample to obtain a very accurate stellar mass.

$\rightarrow$ A different approach to the mass modelling of the cluster could be made using the proper motions of stars in the system, this in principle could be simpler since it avoids some of the mathematical treatment that has to be done for the projected velocity dispersion. The use of the proper motions of stars could be used to measure the velocity dispersion $\sigma$ of the system and with the use of the virial theorem, the dynamical mass of the system could be calculated in a more straightforward way.

$\rightarrow$ As mentioned in chapter 3, the use of integrated spectra of the whole system could be used to determine its dynamical mass by using cross correlation techniques with a template stellar spectrum using the iraf package FXCOR. This procedure is made by measuring the FWHM of the cross correlation results and using the virial theorem (equations 3.2 and 3.3) to get the dynamical mass of the cluster. By getting good quality integrated spectra of $\omega$ Centauri and following this procedure, the mass determination would be made by another different method that would allow the model to be more robust with different constraints that improve the results.

 % Conclusions

\begin{thebibliography}{9}
\addcontentsline{toc}{chapter}{Bibliography}


\bibitem{1} 
Michael J. Kurtz, Douglas J. Mink. 
\textit{RVSAO 2.0: Digital Redshifts and Radial Velocities}. 
Harvard-Smithonian Center for Astrophysics, Cambirdge, MA 02138, 1993.

\bibitem{2}
Roueff F., Salati P., Tillet R. 
\textit{The velocity dispersion profile of globular clusters: a closer look}. 
preprint astro-ph/9707174v1.

\bibitem{3} 
Binney J., Tremaine S.. 
\textit{Galactic Dynamics}. 
Princeton University Press, 1994.

\bibitem{4}
http://news.ucsc.edu/2014/11/globular-clusters.html

\bibitem{5}
R. Ibata, C. Nipoti, A. Sollima et. al. 
\textit{Do globular clusters possess Dark Matter halos?}.
MNRAS, Ras 2012.

\bibitem{6}
Joshua J. Adams, Karl Gebhardt, Guillermo A. Blanc et. al. 
\textit{The central Dark Matter distribution of NGC 2976}.
The Astrophysical Journal, 745:92 (17pp), 2012.

\bibitem{7}
P. J. E. Peebles \& R. H. Dicke. 
\textit{Origin of the Globular Star Clusters}.
The Astrophysical Journal, December 1968.

\bibitem{8}
R. G. Gratton, E. Carreta, A. Bragaglia
\textit{Multiple populations in Globular Clusters}.
The Astronomy and Astrophysics Review, 2012.

\bibitem{9}
Richard B. Larson.
\textit{Globular Clusters as Fossils of Galaxy Formation}.
Yale Astronomy Department.

\bibitem{10}
M. E. Sharina, T. H. Puzia, V. L. Afanasiev et. al.
\textit{Globular clusters in low mass galaxies}.
IAU Colloquium No. 198, 2005.

\bibitem{11}
A. Klypin , A. V. Kravtsov et. al. 
\textit{Where Are the Missing Galactic Satellites?} 
The American Astronomical Society, 1999. 

\bibitem{12}
M. Odenkirchen, E. K. Grebel, W. Dehnen et. al.
\textit{The extended tails of palomar 5: A 10$^{\circ}$ Arc of Globular Cluster Tidal Debris}
The American Astronomical Society, 2003.

\bibitem{13}
S. M. Fall \& M. J. Rees.
\textit{A theory for the origin of Globular Clusters}
The Astrophysical Journal, 298: 18-26, 1985.

\bibitem{14}
C. Conroy, A. Loeb \& D. Spergel
\textit{Evidence Against Dark Matter Halos Surrounding the Globular Clusters MGC1 and NGC 2419}
The Astrophysical Journal, October 11$^{th}$ 2009.

\bibitem{15}
S. S. Larse, J. P. Brodie et. al.
\textit{Nitrogen abundances and multiple stellar populations in the Globular Clusters of the Fornax DSPH}
ApJ, accepted (27 Aug 2014).

\bibitem{16}
V. Guglielmo, N. Amoruso, A. Colombo
\textit{Velocity dispersion in Elliptical Galaxies}
The Sky as a laboratory, 2009.

\bibitem{17}
Lars Hernquist
\textit{An analytical model for spherical galaxies and bulges}
ApJ, 356:359-364, 1990 June 20.

\bibitem{18}
URL: \url{http://www.physics.mcmaster.ca/Globular.html}

\bibitem{19}
Dominici, Tania \& Dos Santos, Claudia Lucia et al. 2008
\textit{Pico dos dias observatory and its instrumentation}
Museu de Astronomia e Ciencias Afins Rio de Janeiro, Brazil, 20921-030

\bibitem{20}
Charbonnel C. et al.
\textit{Are there any first-generation stars in globular clusters today?}
Astronomy \& Astrophysics manuscript no. 2014AA569L6. October 16, 2014.

\bibitem{21}
Portegies Zwart, Simon; McMillan, Stephen L. W.; Gieles, Mark.
\textit{Young Massive Star Clusters}
Annual Review of Astronomy and Astrophysics, vol. 48, p.431-493.

\bibitem{22}
Da Costa, G. S. \& Coleman, Matthew G.
\textit{A Spectroscopic Survey for $\omega$ Centauri Members at and beyond the Cluster Tidal Radius}
The Astronomical Journal, Volume 136, Issue 1, pp. 506-517 (2008).

\bibitem{23}
Oregon University database.
URL: \url{http://pages.uoregon.edu/jimbrau/BrauImNew/Chap23/6th/}

\bibitem{24}
Johnson, Christian I.; Pilachowski, Catherine A. et al.
\textit{Fe and Al Abundances for 180 Red Giants in the Globular Cluster Omega Centauri (NGC 5139)}
The Astrophysical Journal, Volume 681, Issue 2, article id. 1505-1523, pp. (2008).

\bibitem{25}
Merritt, David ; Meylan, Georges \& Mayor, Michel.
\textit{The stellar dynamics of omega centauri}
Astronomical Journal v.114, p. 1074-1086 (1997).

\bibitem{26}
Noyola, Eva; Gebhardt, Karl. et al.
\textit{Very Large Telescope Kinematics for Omega Centauri: Further Support for a Central Black Hole}
The Astrophysical Journal Letters, Volume 719, Issue 1, pp. L60-L64 (2010).

\bibitem{27}
Pancino, E.; Galfo, A. et al.
\textit{The Rotation of Subpopulations in $\omega$ Centauri}
The Astrophysical Journal, Volume 661, Issue 2, pp. L155-L158.

\bibitem{28}
Reijns, R. A et al.
\textit{Radial velocities in the globular cluster $\omega$ Centauri}
Astronomy and Astrophysics, Volume 445, Issue 2, January II 2006, pp.503-511.

\bibitem{29}
Sollima, A. et al.
\textit{The non-peculiar velocity dispersion profile of the stellar system $\omega$ Centauri}
Monthly Notices of the Royal Astronomical Society, Volume 396, Issue 4, pp. 2183-2193.

\bibitem{30}
Toshio Tsuchiya; Dana I. Dinescu \& Vladimir I. Korchagin.
\textit{A Capture Scenario for Globular Cluster Omega Centauri}
Astrophys.J. 589 (2003) L29-L32.

\bibitem{31}
Norris, John E.
\textit{The chalenges of Omega Centauri}
American Astronomical Society, AAS Meeting #220, id.302.01.

\bibitem{32}
IRAF software.
\textit{``Image Reduction and Analysis Facility"}
Supported by the National Optical Astronomy Observatories (NOAO).

\end{thebibliography}

\label{Bibliography}
\lhead{\emph{Bibliography}}  % Change the left side page header to "Bibliography"
\bibliographystyle{unsrtnat}  % Use the "unsrtnat" BibTeX style for formatting the Bibliography
\bibliography{Bibliography}  % The references (bibliography) information are stored in the file named "Bibliography.bib"

\end{document} 

%% ----------------------------------------------------------------
% Now begin the Appendices, including them as separate files

%\addtocontents{toc}{\vspace{2em}} % Add a gap in the Contents, for aesthetics

%\appendix % Cue to tell LaTeX that the following 'chapters' are Appendices

%\input{Appendices/AppendixA}	% Appendix Title

%\addtocontents{toc}{\vspace{2em}}  % Add a gap in the Contents, for aesthetics
%\backmatter

%% ----------------------------------------------------------------

 % The End
%% ----------------------------------------------------------------